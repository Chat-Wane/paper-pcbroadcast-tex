
\section{Experimentation}
\label{sec:experimentation}

% \TODO{Define what we want.} We measure the time between the broadcast of a
% message and its delivery by all processes in the network. We expect that our
% measurements slowly increase as the network becomes more dynamic.

In this section, we evaluate the impact of \CBROADCAST on the message
delivery. The experiments run on the \PEERSIM
simulator~\cite{montresor2009peersim} that allows simulations to reach high
scale in terms of number of processes. Our implementation is available on the
Github platform\footnote{\url{http://github.com/chat-wane/peersim-pcbroadcast}}.


\noindent \textbf{Objective:} To observe the delay introduced by \CBROADCAST on
message delivery. We expect the delay to increase as the latency increase.

\noindent \textbf{Description:} We build an overlay network with a topology
close to random graphs using \SPRAY~\cite{nedelec2017adaptive}. Networks
comprises 1k processes. The network is dynamic. Each process' neighborhood $Q$
changes at least once every 60 seconds. Each exchange involve two processes that
both add and remove half of their partial view.  Links are FIFO, bidirectional,
and have transmission delays.\\
We measure the shortest path length from a random set of processes to all other
processes. It represents the time taken by broadcast messages before being
delivered by processes. 

\noindent \textbf{Results:} 


%%% Local Variables:
%%% mode: latex
%%% TeX-master: "../paper"
%%% End:
