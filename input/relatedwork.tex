
\section{Related work}
\label{sec:relatedwork}

This section reviews the related work of logical clocks. It goes from
piggybacking approaches to vector-based approaches. Then, it reviews explicit
dependency tracking and dissemination-based approaches.

\noindent \textbf{Piggybacking
  approaches~\cite{birman1987reliable,hadzilacos1993fault}}. A trivial way to
ensure causal ordering of messages is to piggyback all causally related messages
since the last broadcast message along with the new broadcast message. Even by
piggybacking the identifiers of messages instead of messages themselves, the
broadcast message size may increase quickly depending on the
application. \CBROADCAST does not piggyback all preceding messages in broadcast
messages. However, an accumulation of messages arises during buffering. As
discussed in Section~\ref{subsec:complexity}, we can assume that links quickly
become safe so the buffer size stays small, and we can easily set a threshold on
the buffer size.% (see Algorithm~\ref{algo:boundingbuffer}).

\noindent \textbf{Vector clock
  approaches~\cite{fidge1988timestamps,mattern1989virtual}.}  A vector clock is
a vector of monotonically increasing counters.  It encodes the partial order of
messages using this vector: $VC(m) < VC(m') \implies m \rightarrow m'$.  Before
delivering a message, processes using vector-based broadcast check if the vector
of the message is ready regarding their local vector. If it detects any missing
preceding message, the process delays the delivery.  To implement this
vector-based broadcast
\begin{inparaenum}[(i)]
\item each process must maintain a vector locally;
\item each message must piggyback such vector;
\item there is 1 counter per process that ever broadcast a message.
\end{inparaenum}
To accurately track causality, processes cannot share their entry. To safely
track causality, processes cannot reclaim entries. Hence, even with
\textbf{compaction approaches~\cite{singhal1992efficient}}, the vectors grow
linearly in terms of number of processes that ever broadcast a message.
In~\cite{almeida2008interval}, the complexity is reduced to the actual number of
processes in the network.  Still, these approaches do not scale, particularly in
dynamic networks subject to churn and failures. \\
In comparison to these vector-based approaches, our approach reduces the
generated traffic of causal broadcast by a factor of $N$ in the most common
context where processes have partial knowledge of the network membership. \\
\textbf{Probabilistic approaches~\cite{mostefaoui2017probabilistic}} sacrifices
on causality tracking accuracy: messages may be delivered out of order under a
computable boundary. The size of control information in messages
depends on the desired boundary. \\
% It alleviates messages from the need to piggyback heavy
  % control information~\cite{friedman2004causal}
Unlike vector-based approach, our broadcast cannot state if two messages are
concurrent, accurate causal delivery is a feature provided by default by the
propagation scheme. Once safe, FIFO links deliver message in causal order
without further delay. The speed of delivery is that of FIFO links.


\noindent Explicitly tracking \textbf{semantic dependencies} allows broadcast
protocols to reduce the size of piggybacked control
information~\cite{bailis2013bolton,lloyd2011cops,mukund2014optimized}. For
instance, when Alice comments Bob's picture, everyone must receive the picture
before the comment. The broadcast message only conveys one semantic
dependency. When Alice comments multiple pictures at once, the broadcast message
conveys all dependencies.  Message overhead increases linearly with the number
of semantic dependencies. To track semantic dependencies, causal broadcast
becomes application dependent. Instead our approach remains
application-agnostic. Comments, pictures, etc. are events that relate to all
preceding events. When Alice comments Bob's picture, everyone will receive this
event before the former event and all other preceding events. Whatever the
number of preceding events, broadcast messages only convey constant size control
information.

\noindent Preserving causal order using \textbf{dissemination paths} reduces
generated traffic by keeping message overhead
constant~\cite{bravo2017saturn,friedman2004causal}. State-of-the-art does not
support dynamic systems~\cite{friedman2004causal}, or supports it using
epochs~\cite{bravo2017saturn} that confines usability to small scale systems
where failures are uncommon. In comparison, we designed \CBROADCAST to handle
large and dynamic systems. Our approach provides a lightweight and efficient
mean to reconfigure dissemination paths using local knowledge without impairing
causal order.  Saturn~\cite{bravo2017saturn} along with \CBROADCAST could ease
the former's online changes in configuration, making the former resilient to
failures and topology changes.


% \PAR{Topological approaches.}{A trivial way to ensure causal delivery of
% messages without message overhead consists in building an overlay network shaped
% as a ring: each broadcast message loops once in FIFO order (\REF).
% Unfortunately, maintaining a specific topology can prove costly in dynamic
% networks subject to churn. Our approach relies on a peer-sampling protocol
% \TODO{(Relies on R-broadcast that relies on such psp)} but we make no assumption
% on the built topology. One can choose the most suitable topology depending on
% network settings. For instance, random peer-sampling
% protocols~\cite{jelasity2007gossip} have a low upkeep and guarantee a connected
% network even in dynamic settings.}

% \TODO{Explain super-peer approaches.}

% \TODO{Inter-group broadcast is a kind of topological approach.}

% \PAR{Inter-group
%   broadcast~\cite{johnson1998scalable,johnson1999intergroup}.}{Inter-group
% broadcast allows processes to broadcast messages to members of several
% inter-connected networks. Paper~\cite{johnson1999intergroup} states that
% inter-group causal broadcast is ensured when groups that internally ensure
% causal broadcast are linked together by communication channels that ensure FIFO
% ordering of messages. One can employ these approaches to cluster the network in
% subsystems. Control information in messages depends of the size of each small
% subsystem. Our approach is an extreme case where each process is a
% subsystem.}

%% The union of paths taken by broadcast messages builds a diffusion tree, i.e.,
%% a directed acyclic graph, that may change at each broadcast.

% \paragraph{Push-pull~\cite{mercier2017optimal}.} In random graphs, pull is
% optimally fast. Push phase of push-pull improves the startup phase of message
% dissemination. We only present push but it easily can be push-pull. \TODO{Maybe
%   too linked to random graphs.}


%%% Local Variables:
%%% mode: latex
%%% TeX-master: "../paper"
%%% End:
