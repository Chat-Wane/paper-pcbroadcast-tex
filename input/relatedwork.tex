
\section{Related work}
\label{sec:relatedwork}

\TODO{Table comparing the improvement. Local overhead. Number of
  messages. Message overhead.}

\paragraph{Piggybacking messages.} A trivial way to ensure causal ordering of
messages is to piggyback all causally related messages along with the newly
broadcast message.


\paragraph{Vector clock-based approaches.} +compaction.
\paragraph{Topological constraint-based approaches.}

\paragraph{Inter-group
  broadcast~\cite{johnson1998scalable,johnson1999intergroup}.} Inter-group
broadcast allows processes to broadcast messages to members of several
inter-connected networks. Paper~\cite{johnson1999intergroup} states that
inter-group causal broadcast is ensured when groups that internally ensure
causal broadcast are linked together by communication channels that ensure FIFO
ordering of messages. Our approach is an extreme case where each process is a
group connected to other groups via FIFO channels. The union of paths taken by
broadcast messages builds a diffusion tree, i.e., an acyclic graph, that may
change at each broadcast.

\paragraph{Already received.} Causal ordering and detecting duplicated receipts
are orthogonal problems. In this paper, we do not provide an implementation for
the later. The simplest approach consists in saving all received messages
(\REF). However, the size of this set linearly and monotonically increases as
the number of broadcast messages increases. One would prefer an approach based
on vectors where one entry corresponds to the number of messages received by a
particular process (\REF). Such approach do not require to piggyback additional
data in the message. However, it requires to store locally a vector the size of
which increases linearly compared to the number of processes that ever broadcast
a message. Interval tree clocks (\REF) allow processes to reduce this
complexity. It becomes linear in terms of number of processes that are currently
involved in broadcasting. Possible improvements could take advantage of the fact
that the number of duplicates is equal to the number of incoming links. However,
it does not hold in dynamic networks where additional links are
established. Finding a sublinear bound for detecting duplicated receipts remains
an open problem.

%%% Local Variables:
%%% mode: latex
%%% TeX-master: "../paper"
%%% End:
