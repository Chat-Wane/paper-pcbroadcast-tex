
\begin{figure*}
  \begin{center}
    \subfloat[Part A][\label{fig:generalsolveA}Process~A broadcasts $a$.]
    {
\begin{tikzpicture}[scale=1]
  
  \small
  
  \newcommand\X{210/5pt};
  \newcommand\Y{30pt};
  
  % \draw[->] ( -0.5*\X, 0.5*\Y) -- ( -5+0*\X, 0*\Y);
  % \draw[->] ( -0.5*\X, 0*\Y) -- ( -5+0*\X, 0*\Y);
  % \draw[->] ( -0.5*\X, -0.5*\Y) -- ( -5+0*\X, 0*\Y);  

  \draw[fill=white] (0*\X, 0*\Y) node{\textbf{A}} +(-5pt, -5pt) rectangle +(5pt, 5pt);
  \draw[fill=white] (1*\X, -1*\Y) node{\textbf{B}} +(-5pt, -5pt) rectangle +(5pt, 5pt);
  \draw[fill=white] (2*\X,  0*\Y) node{\textbf{C}} +(-5pt, -5pt) rectangle +(5pt, 5pt);
  \draw (1*\X, 5-1*\Y) node[above]{$a$};

  \draw[->](5+0*\X, 0*\Y) -- (-5+1*\X, -1*\Y); %% A->B
  \draw[<-](5+0*\X, -5+0*\Y) -- node[sloped, below]{$a\,b$} (-5+1*\X, -5-1*\Y); %% A<-B

  \draw[->](5+0*\X, 5+0*\Y) -- node[above]{$a$} (-5+2*\X, 5+0*\Y); % A->C
  \draw[<-](5+0*\X,  1.25+ 0*\Y) -- (-5+2*\X,  1.25+ 0*\Y); % A<-C
 
  \draw[<-](5+1*\X, -1*\Y) -- (-5+2*\X, 0*\Y); %% B<-C
  \draw[->](5+1*\X, -5-1*\Y) --  node[sloped, below]{$b\,a$} (-5+2*\X, -5+0*\Y); %% B->C

  % \draw[->](5+2*\X, 0*\Y) -- ( 2.5*\X, 0*\Y);
\end{tikzpicture}}
    \hspace{40pt}
    \subfloat[Part B][\label{fig:generalsolveB}Process~B receives and 
    delivers $a$. Process~B forwards $a$ using its FIFO links.]
    {
\begin{tikzpicture}[scale=0.9]
  
  \small
  
  \newcommand\X{210/5pt};
  \newcommand\Y{30pt};
  
  \draw (-1*\X, 0*\Y);
  \draw (3*\X, 0*\Y);

  % \draw[->] ( -0.5*\X, 0.5*\Y) -- ( -5+0*\X, 0*\Y);
  % \draw[->] ( -0.5*\X, 0*\Y) -- ( -5+0*\X, 0*\Y);
  % \draw[->] ( -0.5*\X, -0.5*\Y) -- ( -5+0*\X, 0*\Y);  

  \draw[fill=white] (0*\X, 0*\Y) node{\textbf{A}} +(-5pt, -5pt) rectangle +(5pt, 5pt);
  \draw[fill=white] (1*\X, -1*\Y) node{\textbf{B}} +(-5pt, -5pt) rectangle +(5pt, 5pt);
  \draw[fill=white] (2*\X,  0*\Y) node{\textbf{C}} +(-5pt, -5pt) rectangle +(5pt, 5pt);
  \draw (1*\X, 5-1*\Y) node[above]{$a$};

  \draw[->](5+0*\X, 0*\Y) -- (-5+1*\X, -1*\Y); %% A->B
  \draw[<-](5+0*\X, -5+0*\Y) -- node[sloped, below right]{$\,\,a$} (-5+1*\X, -5-1*\Y); %% A<-B

  \draw[->](5+0*\X, 5+0*\Y) -- node[above left]{$a$} (-5+2*\X, 5+0*\Y); % A->C
  \draw[<-](5+0*\X,  1.25+ 0*\Y) -- (-5+2*\X,  1.25+ 0*\Y); % A<-C
 
  \draw[<-](5+1*\X, -1*\Y) -- (-5+2*\X, 0*\Y); %% B<-C
  \draw[->](5+1*\X, -5-1*\Y) --  node[sloped, below left]{$a\,\,$} (-5+2*\X, -5+0*\Y); %% B->C

  % \draw[->](5+2*\X, 0*\Y) -- ( 2.5*\X, 0*\Y);
\end{tikzpicture}}
    \hspace{40pt}
    \subfloat[Part C][\label{fig:generalsolveC}Process~B broadcasts $b$.]
    {
\begin{tikzpicture}[scale=1]
  
  \small
  
  \newcommand\X{210/5pt};
  \newcommand\Y{30pt};
  
  % \draw[->] ( -0.5*\X, 0.5*\Y) -- ( -5+0*\X, 0*\Y);
  % \draw[->] ( -0.5*\X, 0*\Y) -- ( -5+0*\X, 0*\Y);
  % \draw[->] ( -0.5*\X, -0.5*\Y) -- ( -5+0*\X, 0*\Y);  

  \draw[fill=white] (0*\X, 0*\Y) node{\textbf{A}} +(-5pt, -5pt) rectangle +(5pt, 5pt);
  \draw[fill=white] (1*\X, -1*\Y) node{\textbf{B}} +(-5pt, -5pt) rectangle +(5pt, 5pt);
  \draw[fill=white] (2*\X,  0*\Y) node{\textbf{C}} +(-5pt, -5pt) rectangle +(5pt, 5pt);
%  \draw (1*\X, 5-1*\Y) node[above]{$a$};

  \draw[->](5+0*\X, 0*\Y) -- (-5+1*\X, -1*\Y); %% A->B
  \draw[<-](5+0*\X, -5+0*\Y) -- node[sloped, below]{$a\,\,\,\,\,\,b$}
  (-5+1*\X, -5-1*\Y); %% A<-B

  \draw[->](5+0*\X, 5+0*\Y) -- node[above]{$a$} (-5+2*\X, 5+0*\Y); % A->C
  \draw[<-](5+0*\X,  1.25+ 0*\Y) -- (-5+2*\X,  1.25+ 0*\Y); % A<-C
 
  \draw[<-](5+1*\X, -1*\Y) -- (-5+2*\X, 0*\Y); %% B<-C
  \draw[->](5+1*\X, -5-1*\Y) --  node[sloped, below]{$b\,\,\,\,\,\,a$} (-5+2*\X, -5+0*\Y); %% B->C

  % \draw[->](5+2*\X, 0*\Y) -- ( 2.5*\X, 0*\Y);
\end{tikzpicture}}
    \\
    \subfloat[Part D][\label{fig:generalsolveD}Process~A receives and 
    delivers $b$. Process~A forwards $b$ using its FIFO links.]
    {
\begin{tikzpicture}[scale=0.9]
  
  \small
  
  \newcommand\X{210/5pt};
  \newcommand\Y{30pt};
  
  \draw (-1*\X, 0*\Y);
  \draw (3*\X, 0*\Y);

  % \draw[->] ( -0.5*\X, 0.5*\Y) -- ( -5+0*\X, 0*\Y);
  % \draw[->] ( -0.5*\X, 0*\Y) -- ( -5+0*\X, 0*\Y);
  % \draw[->] ( -0.5*\X, -0.5*\Y) -- ( -5+0*\X, 0*\Y);  

  \draw[fill=white] (0*\X, 0*\Y) node{\textbf{A}} +(-5pt, -5pt) rectangle +(5pt, 5pt);
  \draw (-5 + 0*\X, 0*\Y) node[left]{$b$};
  \draw[fill=white] (1*\X, -1*\Y) node{\textbf{B}} +(-5pt, -5pt) rectangle +(5pt, 5pt);
  \draw[fill=white] (2*\X,  0*\Y) node{\textbf{C}} +(-5pt, -5pt) rectangle +(5pt, 5pt);
%  \draw (1*\X, 5-1*\Y) node[above]{$a$};

  \draw[->](5+0*\X, 0*\Y) --
  node[sloped, above left]{$b\,\,\,$}
  (-5+1*\X, -1*\Y); %% A->B
  \draw[<-](5+0*\X, -5+0*\Y) -- (-5+1*\X, -5-1*\Y); %% A<-B

  \draw[->](5+0*\X, 5+0*\Y) node[above right]{$b$}-- node[above right]{$a$} (-5+2*\X, 5+0*\Y); % A->C
  \draw[<-](5+0*\X,  1.25+ 0*\Y) -- (-5+2*\X,  1.25+ 0*\Y); % A<-C
 
  \draw[<-](5+1*\X, -1*\Y) -- (-5+2*\X, 0*\Y); %% B<-C
  \draw[->](5+1*\X, -5-1*\Y) --  node[sloped, below]{$b\,\,\,\,\,\,a$} (-5+2*\X, -5+0*\Y); %% B->C

  % \draw[->](5+2*\X, 0*\Y) -- ( 2.5*\X, 0*\Y);
\end{tikzpicture}}
    \hspace{40pt}
    \subfloat[Part E][\label{fig:generalsolveE}Process~C cannot receive $b$
    without having received $a$ beforehand.]
    {
\begin{tikzpicture}[scale=1]
  
  \small
  
  \newcommand\X{210/5pt};
  \newcommand\Y{30pt};
  
  % \draw[->] ( -0.5*\X, 0.5*\Y) -- ( -5+0*\X, 0*\Y);
  % \draw[->] ( -0.5*\X, 0*\Y) -- ( -5+0*\X, 0*\Y);
  % \draw[->] ( -0.5*\X, -0.5*\Y) -- ( -5+0*\X, 0*\Y);  

  \draw[fill=white] (0*\X, 0*\Y) node{\textbf{A}} +(-5pt, -5pt) rectangle +(5pt, 5pt);
  \draw[fill=white] (1*\X, -1*\Y) node{\textbf{B}} +(-5pt, -5pt) rectangle +(5pt, 5pt);
  \draw[fill=white] (2*\X,  0*\Y) node{\textbf{C}} +(-5pt, -5pt) rectangle +(5pt, 5pt);

  \draw[->](5+0*\X, 0*\Y) -- (-5+1*\X, -1*\Y); %% A->B
  \draw[<-](5+0*\X, -5+0*\Y) -- (-5+1*\X, -5-1*\Y); %% A<-B

  \draw[->](5+0*\X, 5+0*\Y) -- (-5+2*\X, 5+0*\Y) node[above left]{$b\,a\,\,$}; % A->C
  \draw[<-](5+0*\X,  1.25+ 0*\Y) -- (-5+2*\X,  1.25+ 0*\Y); % A<-C
 
  \draw[<-](5+1*\X, -1*\Y) -- (-5+2*\X, 0*\Y); %% B<-C
  \draw[->](5+1*\X, -5-1*\Y) -- node[sloped, below right]{$\,\,\,\,\,b\,a$} (-5+2*\X, -5+0*\Y); %% B->C

  % \draw[->](5+2*\X, 0*\Y) -- ( 2.5*\X, 0*\Y);
\end{tikzpicture}}
    \caption{\label{fig:generalsolve}Causal broadcast~\cite{friedman2004causal}
      ensures causal order.}
  \end{center}
\end{figure*}


\section{Background and motivations}
\label{sec:motivations}

Causal broadcast ensures that all connected processes deliver each broadcast
message exactly once~\cite{hadzilacos1994modular} following the happen before
relationship~\cite{lamport1978time}. If the sending of a message $m$ precedes
the sending of a message $m'$ then all processes that deliver these two messages
need to deliver $m$ before $m'$. Otherwise they can deliver them in any order.

Encoding the logical time at broadcast regarding all other broadcasts and
piggyback this control information in each broadcast message allow processes to
ensure causal order on message delivery. Instead, \cite{friedman2004causal}
%  checking repeatedly if messages are ready to be delivered, a
% preventive approach~\cite{friedman2004causal} makes sure that messages arrive
%ready in the first place. % It
uses FIFO links and systematically forwards delivered messages.  Intuitively,
the dissemination pattern automatically makes sure that no paths from a process
to another process carry messages out of causal order.

Figure~\ref{fig:generalsolve} depicts this principle. The system comprises 3
processes connected to each other with FIFO links.  In
Figure~\ref{fig:generalsolveA}, Process~A broadcasts $a$. It sends $a$ to
Process~B and Process~C. In Figure~\ref{fig:generalsolveB}, Process~B receives,
delivers, and forwards $a$. In Figure~\ref{fig:generalsolveC}, it broadcasts
$b$. Consequently, all processes must deliver $a$ before delivering $b$. In
Figure~\ref{fig:generalsolveD}, Process~A receives, delivers, and forwards
$b$. Process~A fulfills the causal order constraint between $a$ and $b$. In
Figure~\ref{fig:generalsolveE}, we see that either directly via Process~B or
indirectly via Process~A, Process~C cannot receive $b$ before $a$. Thus, it
eventually receives, delivers, and forwards the messages following causal order.

\begin{figure}
  \begin{center}
    
\begin{tikzpicture}[scale=1]

  \small
  
  \newcommand\X{210/5pt};
  \newcommand\Y{30pt};
  
  \draw[->] ( -1*\X, 1*\Y) -- ( -5+0*\X, 0*\Y);
  \draw[->] ( -1*\X, 0*\Y) node[anchor=east, align=center]{Rest\\of the\\network} -- ( -5+0*\X, 0*\Y);
  \draw[->] ( -1*\X, -1*\Y) -- ( -5+0*\X, 0*\Y);

  \draw[<-] ( 5*\X, 1*\Y) -- ( 5+4*\X, 0*\Y);
  \draw[<-] ( 5*\X, 0*\Y) node[anchor=west, align=center]{Rest\\of the\\network}-- ( 5+4*\X, 0*\Y);
  \draw[<-] ( 5*\X, -1*\Y) -- ( 5+4*\X, 0*\Y);


  \draw[fill=white] (0*\X, 0*\Y) node{\textbf{A}} +(-5pt, -5pt) rectangle +(5pt, 5pt);

  \draw[fill=white] (1*\X, -1*\Y) node{\textbf{B}} +(-5pt, -5pt) rectangle +(5pt, 5pt);

  \draw[fill=white] (2*\X, -2*\Y) node{\textbf{E}} +(-5pt, -5pt) rectangle +(5pt, 5pt);
  \draw[fill=white] (2*\X,  0*\Y) node{\textbf{D}} +(-5pt, -5pt) rectangle +(5pt, 5pt);
  \draw[fill=white] (2*\X,  2*\Y) node{\textbf{C}} +(-5pt, -5pt) rectangle +(5pt, 5pt);

  \draw[fill=white] (4*\X, 0*\Y) node{\textbf{F}} +(-5pt, -5pt) rectangle +(5pt, 5pt);

  \draw[->](5+0*\X, 0*\Y) -- node[sloped, above]{\COLOR{$a'\,a$}} (-5+2*\X,  2*\Y); %% A->E
  \draw[->](5+0*\X, 0*\Y) -- node[sloped, above]{\COLOR{$a'\,a$}} (-5+1*\X, -1*\Y); %% A->B

  \draw[->](5+1*\X, -1*\Y) -- node[sloped, above]{$\COLOR{a'}\,b\,\COLOR{a}$} (-5+2*\X, 0*\Y); %% B->D
  \draw[->](5+1*\X, -1*\Y) -- node[sloped, above]{$\COLOR{a'}\,b\,\COLOR{a}$} (-5+2*\X, -2*\Y); %% B->C

  \draw[->](5+2*\X,  2*\Y) -- node[sloped, above]{$c\,\COLOR{a'\,a}$} (-5+4*\X, 0*\Y); %% E->F
  \draw[->](5+2*\X,  0*\Y) -- node[sloped, above]{$\COLOR{a'}\,b\,\COLOR{a}\,d$} (-5+4*\X, 0*\Y); %% D->F
  \draw[->](5+2*\X, -2*\Y) -- node[sloped, above]{$\COLOR{a'}\,b\,\,e'\,e\,\COLOR{a}$} (-5+4*\X, 0*\Y); %% C->F

  


\end{tikzpicture}
    \caption{\label{fig:disseminationtree}The principle of
      \cite{friedman2004causal} works in large systems where processes have
      partial knowledge of the membership.}
  \end{center}
\end{figure}



In large systems comprising from hundreds to millions of processes, no process
can afford to maintain the full membership to communicate with. Instead,
processes have a much smaller view called neighborhood. Forwarding messages
allows them to reach all members of the system, either directly or transitively
in a gossip fashion~\cite{demers1987epidemic},~\cite{birman1999bimodal}. In
large systems, forwarding is mandatory.  Processes pay the price of gossiping
whatever broadcast protocol. They must create and send copies of the original
broadcast message. Since gossiping already encompasses forwarding of messages,
it does not constitute an additional overhead of~\cite{friedman2004causal}.

Figure~\ref{fig:disseminationtree} shows that such causal broadcast ensures
causal order in larger systems where processes have limited knowledge of the
membership.  Process~A only knows about Process~B and Process~C.  Yet,
Process~A's broadcast messages $a$ and $a'$ arrive to all other processes either
directly or transitively. In addition, $a$ and $a'$ always arrive in causal
order at all processes despite concurrency and whatever the dissemination path.

\begin{figure*}
  \begin{center}
    \subfloat[Part 1][\label{fig:preventiveproblemA}Process~A broadcasts $a$.]
    {
\begin{tikzpicture}[scale=1]
  
  \small
  
  \newcommand\X{210/5pt};
  \newcommand\Y{30pt};
  
  \draw[->] ( -0.5*\X, 0.5*\Y) -- ( -5+0*\X, 0*\Y);
  \draw[->] ( -0.5*\X, 0*\Y) -- ( -5+0*\X, 0*\Y);
  \draw[->] ( -0.5*\X, -0.5*\Y) -- ( -5+0*\X, 0*\Y);  

  \draw[fill=white] (0*\X, 0*\Y) node{\textbf{A}} +(-5pt, -5pt) rectangle +(5pt, 5pt);
  \draw[fill=white] (1*\X, -1*\Y) node{\textbf{B}} +(-5pt, -5pt) rectangle +(5pt, 5pt);
  \draw[fill=white] (2*\X,  0*\Y) node{\textbf{C}} +(-5pt, -5pt) rectangle +(5pt, 5pt);

  \draw[->](5+0*\X, 0*\Y) -- node[sloped, above left]{$a$} (-5+1*\X, -1*\Y); %% A->B
  \draw[->](5+1*\X, -1*\Y) -- (-5+2*\X, 0*\Y); %% B->D

  \draw[->](5+2*\X, 0*\Y) -- ( 2.5*\X, 0*\Y);
\end{tikzpicture}}
    \hspace{20pt}
    \subfloat[Part 2][\label{fig:preventiveproblemA2}Process~A adds a link 
    to Process~D.]
    {
\begin{tikzpicture}[scale=1]
  
  \small
  
  \newcommand\X{210/5pt};
  \newcommand\Y{30pt};
  
  \draw[->] ( -0.5*\X, 0.5*\Y) -- ( -5+0*\X, 0*\Y);
  \draw[->] ( -0.5*\X, 0*\Y) -- ( -5+0*\X, 0*\Y);
  \draw[->] ( -0.5*\X, -0.5*\Y) -- ( -5+0*\X, 0*\Y);  

  \draw[fill=white] (0*\X, 0*\Y) node{\textbf{A}} +(-5pt, -5pt) rectangle +(5pt, 5pt);
  \draw[fill=white] (1*\X, -1*\Y) node{\textbf{B}} +(-5pt, -5pt) rectangle +(5pt, 5pt);
  \draw[fill=white] (2*\X,  0*\Y) node{\textbf{C}} +(-5pt, -5pt) rectangle +(5pt, 5pt);

  \draw[->](5+0*\X, 0*\Y) -- node[sloped, above right]{$a$} (-5+1*\X, -1*\Y); %% A->B
  \draw[->](5+1*\X, -1*\Y) -- (-5+2*\X, 0*\Y); %% B->D

  \draw[->] (5+0*\X, 0*\Y) -- (-5+2*\X, 0*\Y);

  \draw[->](5+2*\X, 0*\Y) -- ( 2.5*\X, 0*\Y);
\end{tikzpicture}}
    \hspace{20pt}
    \subfloat[Part 3][\label{fig:preventiveproblemB}Process~A broadcasts $a'$.]
    {
\begin{tikzpicture}[scale=1]
  
  \small
  
  \newcommand\X{210/5pt};
  \newcommand\Y{30pt};
  
  \draw[->] ( -0.5*\X, 0.5*\Y) -- ( -5+0*\X, 0*\Y);
  \draw[->] ( -0.5*\X, 0*\Y) -- ( -5+0*\X, 0*\Y);
  \draw[->] ( -0.5*\X, -0.5*\Y) -- ( -5+0*\X, 0*\Y);  

  \draw[fill=white] (0*\X, 0*\Y) node{\textbf{A}} +(-5pt, -5pt) rectangle +(5pt, 5pt);
  \draw[fill=white] (1*\X, -1*\Y) node{\textbf{B}} +(-5pt, -5pt) rectangle +(5pt, 5pt);
  \draw (1*\X, 5-1*\Y) node[anchor=south]{$a$};
  \draw[fill=white] (2*\X,  0*\Y) node{\textbf{C}} +(-5pt, -5pt) rectangle +(5pt, 5pt);

  \draw[->](5+0*\X, 0*\Y) -- node[sloped, above]{$a'$} (-5+1*\X, -1*\Y); %% A->B
  \draw[->](5+1*\X, -1*\Y) -- (-5+2*\X, 0*\Y); %% B->D
  
  \draw[->] (5+0*\X, 0*\Y) -- node[anchor=south]{$a'$} (-5+2*\X, 0*\Y); %% A->B

  \draw[->](5+2*\X, 0*\Y) -- ( 2.5*\X, 0*\Y);
\end{tikzpicture}}
    \hspace{20pt}
    \subfloat[Part 4][\label{fig:preventiveproblemC}Process~D receives and
    delivers $a'$ before $a$. This violates causal order.]
    {
\begin{tikzpicture}[scale=1]
  
  \small
  
  \newcommand\X{210/5pt};
  \newcommand\Y{30pt};
  
  \draw[->] ( -0.5*\X, 0.5*\Y) -- ( -5+0*\X, 0*\Y);
  \draw[->] ( -0.5*\X, 0*\Y) -- ( -5+0*\X, 0*\Y);
  \draw[->] ( -0.5*\X, -0.5*\Y) -- ( -5+0*\X, 0*\Y);  

  \draw[fill=white] (0*\X, 0*\Y) node{\textbf{A}} +(-5pt, -5pt) rectangle +(5pt, 5pt);
  \draw[fill=white] (1*\X, -1*\Y) node{\textbf{B}} +(-5pt, -5pt) rectangle +(5pt, 5pt);
  \draw (1*\X, 5-1*\Y) node[anchor=south]{$a'$};
  \draw[fill=white] (2*\X,  0*\Y) node{\textbf{C}} +(-5pt, -5pt) rectangle +(5pt, 5pt);
  \draw (2*\X, 5-0*\Y) node[anchor=south]{$\pmb{a'}$};

  \draw[->](5+0*\X, 0*\Y) --  (-5+1*\X, -1*\Y); %% A->B
  \draw[->](5+1*\X, -1*\Y) -- node[sloped, above]{$b\,\pmb{a}$} (-5+2*\X, 0*\Y); %% B->D
  
  \draw[->] (5+0*\X, 0*\Y) -- (-5+2*\X, 0*\Y); %% A->B

  \draw[->](5+2*\X, 0*\Y) -- ( 2.5*\X, 0*\Y);

\end{tikzpicture}}
    \hspace{20pt}
    \subfloat[Part 4][\label{fig:preventiveproblemD}Process~D propagates
    the causal order violation.]
    {
\begin{tikzpicture}[scale=1]
  
  \small
  
  \newcommand\X{210/5pt};
  \newcommand\Y{30pt};
  
  \draw[->] ( -0.5*\X, 0.5*\Y) -- ( -5+0*\X, 0*\Y);
  \draw[->] ( -0.5*\X, 0*\Y) -- ( -5+0*\X, 0*\Y);
  \draw[->] ( -0.5*\X, -0.5*\Y) -- ( -5+0*\X, 0*\Y);  

  \draw[fill=white] (0*\X, 0*\Y) node{\textbf{A}} +(-5pt, -5pt) rectangle +(5pt, 5pt);
  \draw[fill=white] (1*\X, -1*\Y) node{\textbf{B}} +(-5pt, -5pt) rectangle +(5pt, 5pt);
  \draw[fill=white] (2*\X,  0*\Y) node{\textbf{C}} +(-5pt, -5pt) rectangle +(5pt, 5pt);
  \draw (2*\X, 5-0*\Y) node[anchor=south]{$b$};

  \draw[->](5+0*\X, 0*\Y) --  (-5+1*\X, -1*\Y); %% A->B
  \draw[->](5+1*\X, -1*\Y) -- node[sloped, above]{$a'$} (-5+2*\X, 0*\Y); %% B->D
  
  \draw[->] (5+0*\X, 0*\Y) -- (-5+2*\X, 0*\Y); %% A->B

  \draw[->](5+2*\X, 0*\Y) -- node[above right]{$\pmb{a\,a'}$} ( 2.5*\X, 0*\Y);

\end{tikzpicture}}
    \caption{\label{fig:preventiveproblem}Causal
      broadcast~\cite{friedman2004causal} may violate causal order in dynamic
      settings.}
  \end{center}
\end{figure*}


Unfortunately, \cite{friedman2004causal} ensures causal order only in static
systems where the membership does not change and no links are added or
removed. These are not practical assumptions.  In practice, processes may join
and leave the system at any time; and processes may reconfigure their
neighborhood at any time~\cite{nedelec2016crate}.
Figure~\ref{fig:preventiveproblem} shows an example of message dissemination in
dynamic settings where causal delivery is violated. In
Figure~\ref{fig:preventiveproblemA}, Process~A broadcasts $a$. It sends $a$ to
all its neighbors. Here, it sends $a$ to Process~B only.  Afterwards, in
Figure~\ref{fig:preventiveproblemA2}, Process~A adds a link to
Process~D. Message $a$ is still traveling. In particular, it did not reach
Process~D yet. In Figure~\ref{fig:preventiveproblemB}, Process~A broadcasts
$a'$. In this example, messages travel faster using the direct link from A to D
than using B as intermediate.  We see in Figure~\ref{fig:preventiveproblemC}
that $a'$ arrives at Process~D before $a$. Figure~\ref{fig:preventiveproblemD}
shows that not only it violates causal delivery but also propagates the
violation to all processes downstream.

The causal broadcast presented in this paper extends~\cite{friedman2004causal}
and solves the causal order violation issue of dynamic systems.
Table~\ref{table:comparison} shows its complexity. Most importantly, message
overhead and delivery execution time remain constant, i.e., our approach is both
lightweight in terms of generated traffic and efficient. The local space
complexity is linear in terms of number of processes that ever broadcast a
message, and awaiting messages.  The local space complexity also comprises a
data structure to ensure causal order. We show in
Algorithm~\ref{algo:boundingbuffer} that the size of this structure can be
bounded even in presence of system failures, such as crashes.  This makes causal
broadcast an affordable and efficient middleware for distributed protocols and
applications even in large and dynamic systems.

The next section describes the proposed causal broadcast. It details its
operation, provides the proofs that it works in both static and dynamic
settings, and shows its complexity analysis.

%%% Local Variables:
%%% mode: latex
%%% TeX-master: "../paper"
%%% End:
