
\section{Issues and motivations}
\label{sec:motivations}

\begin{table}
  \caption{\label{table:comparison} Space and time complexity of causal broadcast protocols. $N$ is the number of processes that ever broadcast a message. $W$ is the number of messages that are received by the process but not yet ready to be delivered.
    $P$ is the number of messages that are received between once and the number of expected copies.
    $B$ is the size of the set of temporary buffers.}
  \newcommand{\cmark}{\ding{51}}%
\newcommand{\xmark}{\ding{55}}%

%\setlength{\tabcolsep}{3pt} % General space between cols (6pt standard)

\begin{tabularx}{1.98\columnwidth}{@{}Xcccc@{}}
%  \toprule
  & \makecell{dynamic systems} & \makecell{message overhead} & \makecell{local space consumption} &  \makecell{delivery execution time} \\ \cmidrule{2-5}
  vector-based~\cite{schwarz1994detecting} & \cmark & $O(N)$ & $O(N+W.N)$ & $O(W.N)$ \\
  FIFO+forward~\cite{friedman2004causal} & \xmark & $O(1)$ & $O(P)$ & $O(1)$ \\ \hline\hline
%  objective & \cmark & $?<O(N)$ & $O(N)\leq \, ?$ & $? \leq O(W.N)$ \\ \hline\hline
  \textbf{this paper} & \textbf{\cmark} & $\mathbf{O(1)}$ & $\mathbf{O(N)}$ & $\mathbf{O(1)}$ \\ 
%  \bottomrule
\end{tabularx}

%%% Local Variables:
%%% mode: latex
%%% TeX-master: "../paper"
%%% End:

\end{table}

Table~\ref{table:comparison} highlights the differences in complexity between
reactive and preventive approaches. It also positions our approach towards
state-of-the-art.

\subsection{Reactive approaches}

First row of Table~\ref{table:comparison} shows a vector-clock-based
representative~\cite{schwarz1994detecting} of reactive approaches. Most
state-of-the-art approaches are
reactive~\cite{almeida2008interval,fidge1988timestamps,hadzilacos1993fault,mattern1989virtual,mostefaoui2017probabilistic,singhal1992efficient}. They
handle dynamic networks, i.e., processes can join or leave the network, and
links can be added or removed at any time. Their operation requires that each
broadcast message conveys control information.

\PAR{Message overhead.}{The size of these control information increases linearly
  with either the number of messages~\cite{hadzilacos1993fault} or the number of
  processes that ever broadcast a
  message~\cite{almeida2008interval,fidge1988timestamps,mattern1989virtual,mostefaoui2017probabilistic,singhal1992efficient}. These
  approaches do not handle large networks. These approaches eventually become
  impracticable in dynamic networks, for processes claim an entry when they join
  that cannot be safely reclaimed when they leave.}

\PAR{Local space.}{Control information becomes useful to check if a message is
  ready to be delivered, i.e., messages that should be delivered beforehand are
  actually delivered. Each process locally maintains a vector representing the
  current state of delivery ($O(N)$).  In case the message is not ready, it goes
  into a buffer along with its control information ($O(W.N)$).}

\PAR{Delivery time.}{Upon each receipt, a process must check again all waiting
  messages $W$. A message is ready if 1 entry only among all $N$ entries is
  incremented. It requires $O(N)$ steps, hence $O(W.N)$ for the whole
  buffer. Applications where efficiency is an important matter (\REF) cannot use
  these approaches, for they slow down over time.}


\begin{figure}
  \begin{center}
    \input{input/figvector.tex}
    \caption{\label{fig:vector}Causal broadcast using vector clocks.}
  \end{center}
\end{figure}


\EXAMPLE{Example of vector clock-based causal
  broadcast.}{Figure~\ref{fig:vector} shows a broadcast protocol that ensures
  causal delivery by using vector clocks. When Process $p_1$ broadcasts its
  message $m$, it increments its entry in its vector clock and overloads the
  message with it: $m_{1,\,0,\,0}$. When Process $p_2$ receives the message, it
  checks if it is ready. It immediately delivers it, for the received vector is
  only 1 increment away from its own vector clock. Then Process $p_2$ broadcasts
  a message $m'$. It increments its own entry and overloads the message with its
  vector $[1,\,1,\,0]$. This acknowledges that the delivery of $m$ precedes the
  broadcast of $m'$. When Process $p_3$ receives the later, it detects that $m'$
  is not ready and delays it until it receives and delivers $m$. Then, $p_3$
  checks $m'$ again. $p_3$ finds that $m'$ is ready and delivers it. Using
  vector clocks, the message delivery order follows causal order.}


\TODO{Conclusion?}


\subsection{Preventive approaches}

Second row of Table~\ref{table:comparison} shows a representative of preventive
approaches~\cite{friedman2004causal}. It clearly outperforms reactive approaches
but only works for static networks. The overhead of messages is constant, for it
only rely on FIFO links. The local structure can be pruned, for the network is
static and the number of copies of a same message is known. Messages arrive
ready so processes deliver them in constant time.

% \PAR{Message overhead.}{This approach relies on FIFO links. Each message must
%   convey only 1 scalar to implement FIFO. Since processes must deliver each
%   message exactly once, each message also convey a unique identifier, e.g.,
%   $\langle process\_id,\, increasing\_counter\rangle$. Hence, $O(1)$.}

% \PAR{Local space.}{Since the network is static, each process knows the number of
%   copies it must receive. It can prune its local structure when it receives the
%   expected number of copies. We cannot achieve better local space complexity.}

% \PAR{Delivery time.}{Received messages are ready when they arrive. Processes
%   only check if the received message is a copy of an already delivered
%   message. It achieves this in constant time.}


\begin{figure}
  \begin{center}
    \input{./input/figstatic.tex}
    \caption{\label{fig:static}Preventive approach in static networks.}
  \end{center}
\end{figure}

\EXAMPLE{Example of preventive broadcast.}{Figure~\ref{fig:static} shows
  a broadcast protocol that ensures causal delivery by using FIFO channels.  The
  network is static and comprises 3 processes $p_1$, $p_2$, $p_3$ linked to each
  other. The figure does not show messages forwarded by $p_3$ for the sake of
  clarity.  The processes receive $m$ and $m'$ multiple times but there exists
  no link in the paths from $p_2$ to $p_3$ that carries $m'$ without having
  carried $m$ beforehand. Hence, the delivery of $m$ always precedes the
  delivery of $m'$ at any process.}


Unfortunately, this protocol ensures causal delivery only in static
networks. This does not hold in dynamic settings where processes can join the
network, or links can be added.

\begin{figure}
  \begin{center}
    \input{./input/figproblem.tex}
    \caption{\label{fig:problem}Preventive approach in dynamic network.}
  \end{center}
\end{figure}

\EXAMPLE{Example of preventive approach in dynamic network.}{
  Figure~\ref{fig:problem} illustrates the issue with the establishment of new
  FIFO channels. In this example, a FIFO channel links $p_1$ to $p_2$; another
  links $p_2$ to $p_3$; none links $p_1$ to $p_3$. Other FIFO channels may exist
  but we do not show them for the sake of simplicity. Process $p_1$ broadcasts
  $m$ and delivers it. $p_3$ receives it by the intermediary of $p_2$. In the
  meantime, $p_1$ creates a FIFO channel to $p_3$, then broadcasts $m'$ to $p_2$
  and $p_3$. Since the path through $p_2$ is longer in terms of propagation time
  compared to the direct connections from $p_1$ to $p_3$, Process $p_3$ receives
  and delivers $m'$ before $m$. It violates causal order, for $m'$ precedes $m$.
}

\subsection{Objective}

\TODO{Meow.}



%%% Local Variables:
%%% mode: latex
%%% TeX-master: "../paper"
%%% End:
