
\section{Introduction}

Causal broadcast~\cite{hadzilacos1994modular} is a fundamental building block of
many distributed applications~\cite{nakamoto2009bitcoin} such as distributed
social networks~\cite{borthakur2013petabyte}, distributed collaborative
software~\cite{nedelec2016crate,heinrich2012exploiting}, or distributed data
stores~\cite{demers1987epidemic,shapiro2011comprehensive,bailis2013bolton,lloyd2011cops,bravo2017saturn}.
% over a message-passing system goes through the definition and the
% implementation of appropriate communication
% primitives.  In this paper, we focus on
% causal broadcast.
Causal broadcast is a reliable broadcast where all connected processes
deliver each broadcast message exactly once following the happen
before relationship~\cite{lamport1978time,schwarz1994detecting}: when
Alice comments Bob's picture, everyone receives the comment after the
picture; unrelated events are delivered in any order.
% If the sending of a message $m$ precedes the sending of a message $m'$ then all
% processes that deliver these two messages need to deliver $m$ before
% $m'$. Otherwise they can deliver them in any order.

Unfortunately, causal broadcast has proven expensive in dynamic environments
where any process can broadcast a message at any
time~\cite{charronbost1991concerning}. While gossiping constitutes an efficient
mean to disseminate messages to millions of
processes~\cite{demers1987epidemic},~\cite{birman1999bimodal}, ensuring causal
delivery of these messages remains overcostly.  Using state-of-the-art
protocols, each message piggybacks a -- possibly compressed -- vector of
Lamport's
clocks~\cite{almeida2008interval,fidge1988timestamps,mattern1989virtual,singhal1992efficient}.
The message overhead increases monotonically, for entries cannot be reclaimed
without consensus. The message overhead increases linearly with the number of
processes $N$ that ever broadcast a message in the system. Several messages $W$
may differ their delivery, for preceding messages did not arrive
yet~\cite{mehdi2017slowdown}.  The delivery execution time takes linear time
$O(W.N)$ as well.  Causal broadcast protocols based on vectors eventually become
overcostly and inefficient.


To provide causal order, \cite{friedman2004causal} employs a different
strategy. Instead of piggybacking a vector in each message, processes forward
all messages exactly once using FIFO communication means. Gossip encompasses
forwarding so this does not constitute an overhead of the approach.  Messages
arrive ready so they are delivered immediately. This approach is both
lightweight and efficient. However, its scope is restricted to static systems.
In dynamic systems where processes can join, leave, add or remove communication
means, using this approach may lead to causal order violations.


In this paper, we break the scalability barrier of causal broadcast for large
and dynamic systems.  Our contribution is threefold:
\begin{itemize}[leftmargin=*]
\item We provide a powerful extension of~\cite{friedman2004causal} that extends
  its scope to dynamic systems. We prove that adding new communication means
  between processes constitutes the sole factor in causal order violation. Our
  approach solves this using bounded buffers and few control messages.
  % Then, we provide the formal proof that our broadcast protocol ensures causal
  % order in dynamic systems.
\item We provide the complexity analysis of our broadcast
  protocol. Table~\ref{table:comparison} compares our protocol to two
  representative solutions. Our approach handles dynamic systems while providing
  constant size overhead on messages, and constant delivery execution time.
\item We provide an experimentation highlighting the impact of our protocol on
  transmission delays before delivery. Indeed, to tolerate dynamicity our
  protocol temporarily disables new communication means.  The experiment shows
  that even under bad network conditions and high dynamicity, our protocol
  hardly degrades the mean transmission time before delivery.
\end{itemize}
Consequently, causal broadcast finally becomes an affordable and efficient
middleware for distributed protocols and applications in large and dynamic
systems.

\begin{table}
  \caption{\label{table:comparison} Complexity of causal
    broadcast protocols. 
    $N$ is the number of processes that ever broadcast a message.
    $W$ is the number of received messages awaiting delivery.
    $P$ is the number of delivered messages that are temporarily kept before 
    being safely purged to forbid double delivery.%
%    $B$ is the number of delivered messages that are temporarily kept to
%    check the safety of new communication means.
  }
  \newcommand{\cmark}{\ding{51}}%
\newcommand{\xmark}{\ding{55}}%

\setlength{\tabcolsep}{3pt} % General space between cols (6pt standard)

\begin{tabularx}{0.98\columnwidth}{@{}Xcccc@{}}
%  \toprule
  & dynamic & message overhead & local space &  delivery time \\ \cmidrule{2-5}
  reactive~\cite{schwarz1994detecting} & \cmark & $O(N)$ & $O(N+W.N)$ & $O(W.N)$ \\
  preventive~\cite{friedman2004causal} & \xmark & $O(1)$ & $O(P)$ & $O(1)$ \\ %\hline
  objective & \cmark & $?<O(N)$ & $O(N)\leq \, ?$ & $? \leq O(W.N)$ \\ \hline\hline
  \textbf{this paper} & \textbf{\cmark} & $\mathbf{O(1)}$ & $\mathbf{O(N+B)}$ & $\mathbf{O(1)}$ \\ 
%  \bottomrule
\end{tabularx}

%%% Local Variables:
%%% mode: latex
%%% TeX-master: "../paper"
%%% End:

\end{table}

The rest of this paper is organized as follows: Section~\ref{sec:motivations}
shows the background and motivations of our work. Section~\ref{sec:proposal}
defines our model, describes our proposal, provides the corresponding proofs,
and details its complexity. Section~\ref{sec:experimentation} explains the
results of experimentation.  Section~\ref{sec:relatedwork} reviews the related
work. We conclude in Section~\ref{sec:conclusion}.

%%% Local Variables:
%%% mode: latex
%%% TeX-master: "../paper"
%%% End:
