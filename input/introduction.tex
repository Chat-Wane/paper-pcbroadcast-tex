
\section{Introduction}


Broadcasting~\cite{hadzilacos1994modular} is a fundamental communication
mechanism for numerous distributed
protocols~\cite{nakamoto2009bitcoin,shapiro2011comprehensive} and
applications~\cite{nedelec2016crate}.  Broadcast protocols ensure that all
connected processes receive and deliver each broadcast message exactly once. In
large scale networks, gossiping constitutes an efficient way to broadcast
messages (\REF): processes build an overlay network by maintaining small local
view of the network membership (\REF) and each broadcast message reach all
processes either directly or transitively via forwarding. These views change
constantly for systems generally run in highly dynamic environments where
processes can join, leave, or self-reconfigure at any
time~\cite{mostefaoui2005static}.

% A dynamic network~\cite{mostefaoui2005static} comprises interconnected
% processes. Processes can join or leave the network at any moment. Processes can
% communicate with each other. Broadcasting is a communication mean where a
% message sent by a process will reach all other processes of the network.
In this paper, we focus on causal broadcast. The order of message delivery
follows the happen before
relationship~\cite{lamport1978time,schwarz1994detecting}. This alleviates
systems from the burden of tracking causality between its operations.
% For instance, distributed data stores such as Riak, Dynamo, or Cassandra (\REF)
% and collaborative editors such as Crate, or Peerpad (\REF) make extensive use of
% conflict-free replicated data
% types~\cite{shapiro2011comprehensive,shapiro2011conflict}. They feature addition
% and deletion of elements. A deletion must follow the corresponding addition, for
% the sake of eventual consistency~\cite{bailis2013eventual}. Causal broadcast
% relieves these applications from continuously checking such constraints.
% For instance, distributed collaborative editors perform deletion immediately,
% for causal broadcast already guaranteed that the message containing the
% corresponding insertion was delivered beforehand.
Unfortunately, accurate causality tracking has proven expensive especially in
large and dynamic networks~\cite{charronbost1991concerning}. Most approaches are
reactive: they check if preceding messages are missing at each receipt and
deliver or delay the message accordingly. They either
\begin{inparaenum}[(i)]
\item piggyback preceding messages in new broadcast
  messages~\cite{birman1987reliable,hadzilacos1993fault},
\item or maintain and send vector clocks along with new broadcast
  messages~\cite{fidge1988timestamps,mattern1989virtual}.
\end{inparaenum}
Approaches reduce the size of vectors either by
\begin{inparaenum}[(a)]
\item maintaining local structures to send the least amount of changing
  data~\cite{singhal1992efficient}, or by
% \item relaxing the causal order using probabilistic
%   structures~\cite{mostefaoui2017probabilistic}, or by
\item constraining the network topology (\REF).
\end{inparaenum}
None of these approaches scales in large and dynamic networks, for
% they overload each message with linearly increasing data
% structures. % that depend of the number of processes or the number of messages.
generated traffic and delivery time of messages increase linearly with the
number of processes.

Causal broadcast can be preventive~\cite{friedman2004causal}: processes receive
messages that are immediately ready for causal delivery. It overloads messages
with constant size control information, for it uses FIFO communication
links. Delivery time of messages is constant, for it only discards message
duplicates. Nothing limits the scalability of the system.  However, these highly
desirable properties only hold in static environments where processes cannot
join nor leave nor self-reconfigure. Any change in network membership may lead
to causal delivery violations. A preventive causal broadcast that generalizes
these complexity to dynamic networks would finally make causal broadcast
\TODO{safe}, affordable and efficient in large and dynamic systems.

In this paper, we break the scalability barriers of causal broadcast for large
and dynamic networks. We propose a uniform reliable broadcast that outperforms
state-of-the-art in size of broadcast messages, execution time complexity, and
local space complexity. Our approach belongs to preventive causal
broadcast. Message overhead is constant. Message delivery time is
constant. Local space complexity is that of reliable broadcast plus buffers of
messages the size of which can be bounded.
%% ensures causal delivery without any message overhead.
% To provide causal order, most state-of-the-art approaches are
% reactive~\cite{almeida2008interval,fidge1988timestamps,hadzilacos1993fault,mattern1989virtual,mostefaoui2017probabilistic,singhal1992efficient},
% for they check if message deliveries should be delayed to avoid causality
% violations. On the opposite our approach is
% preventive~\cite{birman1987reliable,friedman2004causal}, for messages are
% immediately delivered on receipt without risk of causality violations.
We prove that our causal broadcast protocol handles both static and dynamic
networks. \TODO{Experiment.}
% relies on lightweight assumptions but contrarily to the former, we prove that it
% handles both static and dynamic networks.

% It makes use of FIFO channels (e.g. TCP) and relies on a peer-sampling
% protocol~\cite{jelasity2007gossip} ensuring a network without partitions
% \TODO{Specify that R-broadcast needs this and we need R-broadcast}.  Complexity
% trade-offs such as the number of messages sent by each process, or the number of
% hops before a broadcast message reach all processes, are independent of our
% protocol and only depends of the peer-sampling protocol.

The rest of this paper is organized as follows: Section~\ref{sec:motivations}
shows the background and motivations of our
work. % Section~\ref{sec:preliminaries}
% introduces definitions and theorems about causal
% broadcast.
Section~\ref{sec:proposal} defines our model, describes our proposal, provides
the corresponding proofs, and details the complexity
analysis. Section~\ref{sec:experimentation} shows the results of
experimentation.  Section~\ref{sec:relatedwork} reviews the related work. We
conclude in Section~\ref{sec:conclusion}.


% Broadcast is a communication primitive allowing a process to send a message to
% all other processes in the network. We can add order on message delivery. It
% alleviates the burden of applications to check message relationships. FIFO
% broadcast exists but not so useful. Causal broadcast is very useful; many
% applications require causal order. Holy grail is total order broadcast but not
% scalable, equivalent to consensus.

% We are in point-to-point context.

%%% Local Variables:
%%% mode: latex
%%% TeX-master: "../paper"
%%% End:
