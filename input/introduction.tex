
\section{Introduction}


Broadcasting~\cite{hadzilacos1994modular} is a fundamental communication
mechanism for numerous
protocols~\cite{nakamoto2009bitcoin,shapiro2011comprehensive} and
applications~\cite{nedelec2016crate}. Processes can send messages that will
reach all other processes of their network. In this paper, we focus on causal
broadcast that ensures that processes order message delivery following the
happen before relationship~\cite{lamport1978time}. This alleviates systems from
the burden of tracking causality between its operations. For instance,
distributed collaborative editors perform deletion immediately, for causal
broadcast already guaranteed that the message containing the corresponding
insertion was delivered beforehand.

Unfortunately, causality tracking is
expensive~\cite{charronbost1991concerning}. Current approaches either
\begin{inparaenum}[(i)]
\item piggyback preceding messages in new broadcast
  messages~\cite{birman1987reliable,hadzilacos1993fault},
\item or maintain and send vector clocks along with new broadcast
  messages~\cite{fidge1988timestamps,mattern1989virtual}.
\end{inparaenum}
More recent approaches focus on reducing the size of vectors either by
\begin{inparaenum}[(a)]
\item maintaining local structures to send the least amount of changing
  data~\cite{singhal1992efficient}, or by
\item relaxing the causal order using probabilistic
  structures~\cite{mostefaoui2017probabilistic}, or by
\item constraining the network topology (\REF).
\end{inparaenum}
None of these approaches scales in large and dynamic networks subject to churn,
for they overload each message with linearly growing data structures that depend
of the number of processes or the number of messages.

In this paper, we solve the scalability issue of causal broadcast. We propose a
uniform reliable broadcast that ensures causal order on message delivery without
any message overhead.  It makes use of FIFO channels (e.g. TCP) and relies on a
peer-sampling protocol~\cite{jelasity2007gossip} ensuring a network without
partitions.  Complexity trade-offs such as the number of messages sent by each
process, or the number of hops before a broadcast message reach all processes,
are independent of our protocol and only depends of the peer-sampling protocol.

The rest of this paper is organized as follows: Section~\ref{sec:preliminaries}
introduces definitions and theorems about causal
broadcast. Section~\ref{sec:proposal} describes our causal broadcast mechanism
and provides the corresponding proofs. Section~\ref{sec:relatedwork} reviews the
related work. We conclude in Section~\ref{sec:conclusion}.


% Broadcast is a communication primitive allowing a process to send a message to
% all other processes in the network. We can add order on message delivery. It
% alleviates the burden of applications to check message relationships. FIFO
% broadcast exists but not so useful. Causal broadcast is very useful; many
% applications require causal order. Holy grail is total order broadcast but not
% scalable, equivalent to consensus.

% We are in point-to-point context.

%%% Local Variables:
%%% mode: latex
%%% TeX-master: "../paper"
%%% End:
