
\section{Conclusion}
\label{sec:conclusion}

In this paper, we described a uniform reliable causal broadcast protocol that
breaks scalability barriers for large and dynamic systems. It extends preventive
causal broadcast complexity to dynamic systems. Among others, message overhead
and delivery execution time remain constant.
% Its space complexity remains small and can be bounded easily. 
This result means that causal broadcast finally becomes affordable and efficient
in large and dynamic systems.

As future work, we plan to investigate on reducing the space complexity of
reliable broadcast. Section~\ref{subsec:complexity} reviews structures with
linearly increasing space consumption. We can reduce this complexity in static
systems. We can prune the structure from already received messages, for we know
that the number of duplicates is equal to the number of incoming
links. Unfortunately, this does not hold in dynamic systems. We would like to
investigate on a way to prune the structure in such settings. This would make
causal broadcast scalable as well on generated traffic as on space consumption.

We also plan to investigate on retrieving partial order of
events. Section~\ref{sec:relatedwork} states that vector-based approaches allows
to compare an event with any other event. They can decide on whether one
precedes the other, or they are concurrent. They can build the partial order of
event using this knowledge. Preventive approaches cannot by default. However, in
extreme settings where the overlay network is fully connected, we can assign a
vector to each received message using local knowledge only, and without message
overhead. We would like to investigate on a way to build these vectors locally
in more realistic settings where processes have partial knowledge of the
membership. We would like to analyze its minimal cost.

%%% Local Variables:
%%% mode: latex
%%% TeX-master: "../paper"
%%% End:
