
\section{Conclusion}
\label{sec:conclusion}

% In this paper, we described a uniform reliable causal broadcast that removes the
% need for messages to carry any control information. To handle dynamic networks,
% our approach only requires the help of buffers the size of which can be bounded
% easily. \TODO{Meow meow meow.}


In this paper, we described a uniform reliable causal broadcast protocol that
breaks scalability barriers. Its space complexity remains small and can be
bounded easily. It does not overload messages with any control information. This
result means that large and dynamic systems can finally afford causal broadcast.

% This makes our causal broadcast suitable for large scale networks subject to
% churn.

% Applications and protocols such as distributed data stores, distributed
% collaborative editors that need to track causal relationships between operations
% can now rely on our causal broadcast.


As future work, we plan to investigate on reducing the space complexity of
reliable broadcast. Section~\ref{subsec:complexity} reviews structures with
linearly increasing space consumption. We can reduce this complexity in static
networks. We can prune the structure from already received messages, for we know
that the number of duplicates is equal to the number of incoming links. As for
FIFO-based broadcast, this does not hold in dynamic networks. We would like to
investigate on a way to prune the structure in such settings. This would make
causal broadcast fully scalable as well on generated traffic as on space
consumption.

% As future work, we plan to investigate on the problem of duplicated message
% detection. Although it only impacts local space complexity, this is the last and
% only remaining linear complexity.

% A possible way of improvement lies in the fact that, in static network, each
% process knows how many duplicates of each message it will receive. We could use
% this information to prune the structure of already received
% messages. Unfortunately, it does not hold in dynamic networks.


%% It is awesome.



%%% Local Variables:
%%% mode: latex
%%% TeX-master: "../paper"
%%% End:
