
\section{Preliminaries}
\label{sec:preliminaries}

Definitions and theorems come from~\cite{hadzilacos1994modular}.

\begin{definition}[\TODO{Overlay} Network]
  An \TODO{overlay} network $N$ comprises a set of processes $P$. Each Process
  runs a set of
  instructions sequentially. \\
  A network $N$ also comprises a set of links $E: P \times P$. $p$'s
  neighborhood $Q$ is the set of links departing from $p$. Processes can
  communicate with their
  neighbors using messages. \\
  Processes are faulty if they crash, otherwise they are correct. The set of
  correct processes is $C$. There are no byzantine processes.
\end{definition}

\begin{definition}[Static and dynamic networks]
  A network is static if both its set of processes and its set of edges are
  immutable. Otherwise, the network is dynamic.
\end{definition}

For the rest of the paper, we only consider networks without partitions.

\begin{definition}[Network partition]
  A network has partitions if there exist two correct processes without any path
  between them, i.e., without a link or a sequence of links comprising correct
  processes only.
\end{definition}

%\TODO{Replace ``network'' by  ``distributed system'' ?}

We define time in a logical sense using Lamport's
definition~\cite{lamport1978time}.

\begin{definition}[Happen before]
  Happen before is a transitive, irreflexive, and antisymmetric relation that
  defines a strict partial orders of events. At process $p$, Event $e$ happens
  before -- or precedes -- Event $e'$ is noted $e_p \rightarrow e'_p$. The
  sending of a message $s_p(m)$ always precedes its receipt $r_q(m)$: \\
  $\forall p,\,q \in P, s_p(m) \rightarrow r_q(m)$.
\end{definition}

Processes communicate by sending messages to other processes. They can send
messages to specific processes or all of them.

\begin{definition}[Uniform reliable broadcast]
  A process $p$ can broadcast a message $b_p(m)$, receive a message $r_p(m)$,
  and deliver a message $d_p(m)$.  When a process $p$ broadcasts a message $m$
  to all processes of the network, correct processes eventually receive it: 
  $\forall p \in P,\, (b_p(m) \Leftrightarrow \forall q \in P,\, r_q(m))$. \\
  Uniform reliable broadcast guarantees 3 properties: \\
  \textbf{Validity:} If a correct process broadcasts a message, then it
  eventually
  delivers it: $\forall p \in C,\, b_p(m) \rightarrow d_p(m)$. \\
  \textbf{Uniform Agreement:} If a process -- correct or not -- delivers a
  message,
  then all correct processes eventually deliver it:\\
  $\forall p \in P,\, (d_p(m) \implies \forall q \in C,\, d_q(m))$. \\
  \textbf{Uniform Integrity:} A process delivers a message at most once, and
  only if it was previously broadcast:\\
  $\forall p \in P,\, \neg(d_p(m) \rightarrow d_p(m)) \wedge$\\$d_p(m) \implies
  \exists q \in P,\, b_q(m) \rightarrow d_p(m)$.
\end{definition}

\begin{algorithm}[h]
  \SetKwProg{Function}{function}{}{}
\SetKwProg{Upon}{upon}{}{}
\SetKwProg{Initially}{INITIALLY:}{}{}
\SetKwProg{Dissemination}{DISSEMINATION:}{}{}

\small

\DontPrintSemicolon
\LinesNumbered

\Initially {} {
  $Q$ \tcp*{$p$'s neighborhood}
  $received \leftarrow \varnothing$ \tcp*{To detect double receipts}
}

\BlankLine

\Dissemination{}{
  
  \Function{\textsc{R-broadcast}($m$) \tcp*[f]{$b_p(m)$}} { 
    $received \leftarrow received \cup m$ \;
    \lForEach {$q \in Q$} {\textsc{sendTo}($q,\, m$)}
    \textsc{R-deliver}($m$) \tcp*{$d_p(m)$}
  }

  \BlankLine
  
  \Upon{\textsc{receive}($m$)}{
    \If {$m \not \in received$} {
      $received \leftarrow received \cup m$ \;
      \lForEach {$q \in Q$} {\textsc{sendTo}($q,\, m$) \tcp*[f]{forward}}
      \textsc{R-deliver}($m$) \tcp*{$d_p(m)$}
    }
  }
  
}


%%% Local Variables:
%%% mode: latex
%%% TeX-master: "../paper"
%%% End:

  \caption{\label{algo:reliablebroadcast}R-broadcast at Process $p$.}
\end{algorithm}

Algorithm~\ref{algo:reliablebroadcast} shows the instructions of a uniform
reliable broadcast. It uses a structure that keeps track of received messages in
order to deliver them at most once. 
%It uses a peer-sampling protocol that
%provides neighbors to communicate with, i.e., a set of links. 
%%Assuming a network without partitions meaning that there exists at least one
%path from any process to any correct process, then all correct processes
%eventually receive all messages at least once:
Since the network does not have partitions, processes either receive the message
directly from the broadcaster or transitively. Thus, all correct processes
eventually deliver all messages exactly once. This algorithm ensures validity,
uniform agreement, and uniform integrity.

In addition to reliably conveying messages to all correct processes, broadcast
protocols can ensure that messages are delivered in a specific order.

To order messages broadcast from one process, we define FIFO order.

\begin{definition}[FIFO order]
  If a process broadcasts two messages, processes deliver the first before the
  second:\\
  $\forall p,\,q \in P,\,$\\$b_p(m) \rightarrow b_p(m') \implies d_q(m) \rightarrow
  d_q(m')$.
\end{definition}

To order messages broadcast by different processes, we define local order.

\begin{definition}[Local order]
  If a process broadcasts a message after having delivered another message
  broadcast by another process, processes deliver the later before the former:\\
  $\forall p,\,q,\,r,\, \in P,\,p\neq q,\,$\\$b_p(m) \wedge d_q(m) \rightarrow b_q(m') \implies d_r(m) \rightarrow d_r(m')$.
\end{definition}

To order messages broadcast by every processes, we define causal order.

\begin{definition}[Causal order]
  The delivery order of messages follows the happen before relationships of the
  corresponding broadcasts:\\ $\forall
  p,\,q,\,r \in P,\,$\\$b_p(m) \rightarrow b_q(m') \implies d_r(m) \rightarrow d_r(m')$.
\end{definition}

\begin{theorem}[\label{theo:causal}Causal order equivalence]
%  The intersection of 
  FIFO order and local order is equivalent to causal order.
\end{theorem}

\begin{definition}[Causal broadcast]
  Causal broadcast is a uniform reliable broadcast ensuring causal order.
\end{definition}

% Multiple approaches achieve causal delivery by piggybacking control information
% in broadcast messages (\REF). Upon receipt of messages, processes check if the
% message is ready to be delivered or must be delayed until preceding messages 
% arrive. \TODO{Should I briefly state the limitations? (complexity).}



% Another approach achieves causal delivery without overloading broadcast messages
% with control information (\REF). It only assumes FIFO links and deterministic
% overlay network (\REF). Processes deliver messages as soon as they arrive.

% \begin{definition}[FIFO link]
%   If Process $p$ sends a message $m$ then $m'$ to Process $q$ using a FIFO link,
%   $q$ receives $m$ before $m'$:
%   $s_p(m) \rightarrow s_p(m') \implies r_q(m) \rightarrow r_q(m')$
% \end{definition}


In this paper, we introduce a causal broadcast protocol that
\begin{inparaenum}[(i)]
\item does not convey any control information in broadcast messages, and 
\item handles both static and dynamic networks.
\end{inparaenum}


%%% Local Variables:
%%% mode: latex
%%% TeX-master: "../paper"
%%% End:
