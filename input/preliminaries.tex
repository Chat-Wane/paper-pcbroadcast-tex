
\section{Preliminaries}
\label{sec:preliminaries}

\begin{definition}[Process]
  is said correct if it does not crash.
\end{definition}

\begin{definition}[Broadcast]
  
\end{definition}

\begin{definition}[Uniform reliable broadcast]

\begin{asparadesc}
\item [Validity:] If a correct process broadcasts $m$, then it eventually delivers $m$.
\item [Uniform Agreement:] If a process -- correct or not -- delivers $m$, then
  all correct processes eventually deliver $m$.
\item [Uniform Integrity:] $m$ is delivered by a process at most once, and only
  if it was previously broadcast.
\end{asparadesc}

\end{definition}


\begin{definition}[Causal delivery]
  
\end{definition}

\begin{figure}
  \begin{center}
  
\begin{tikzpicture}[scale=1]
  \small

  \newcommand\X{35pt};
  \newcommand\Y{30pt};
  
  \draw[->](0pt,   0pt)--(6*\X,   0pt);
  \draw[->](0pt, -1*\Y)--(6*\X, -1*\Y);
  \draw[->](0pt, -2*\Y)--(6*\X, -2*\Y);
  
  \draw[fill=black](0pt, 0pt) node[anchor=east]{$p_1$}circle(2pt);
  \draw[fill=black](0pt, -1*\Y) node[anchor=east]{$p_2$}circle(2pt);
  \draw[fill=black](0pt, -2*\Y) node[anchor=east]{$p_3$}circle(2pt);

  \draw[->](0.5 * \X, 0pt)node[anchor=south]{$ins(a_{1,\,1})$}--(1*\X,   -1 * \Y);
%  \draw[->](0.5 * \X, 0pt)-- (2*\X, -1*\Y) -- ( 5 * \X,   -2 * \Y);
  \draw[->](0.5 * \X, 0pt) to[out=-45, in=90+60] ( 5 * \X,   -2 * \Y);
  \draw(1.5*\X, -2pt)--node[anchor=south]{$\{a_{1,\,1} \}$}(1.5*\X, 2pt);
  \draw(1.25 * \X, -2 -1*\Y )--node[anchor=north]{$\{a_{1,\,1} \}$}(1.25 * \X,    2 -1 * \Y);
  \draw(5.3 * \X, -2 -2*\Y )--node[anchor=north]{$\{a_{1,\,1}\}$}(5.3 * \X,   2 -2 * \Y);

  \draw[->](3.5 * \X, -1*\Y )--(4 * \X,   -2 * \Y);
  \draw[->](3.5 * \X, -1*\Y )node[anchor=south east]{$rem(a_{1,\,1})$}--(4 * \X,    0 * \Y);
  \draw(4.5 * \X, -2 -0*\Y )--node[anchor=south]{$\{\,\}$}(4.5 * \X,    2 -0 * \Y);
  \draw(4.5 * \X, -2 -1*\Y )--node[anchor=south]{$\{\,\}$}(4.5 * \X,    2 -1 * \Y);
  \draw(4.5 * \X, -2 -2*\Y )--node[anchor=north]{$\{\,\}$}(4.5 * \X,    2 -2 * \Y);

  % \draw[decorate,decoration={brace,amplitude=6pt,mirror,raise=4pt}]
  % (4.5*\X, -2.5*\Y) -- node[anchor=north, yshift=-10pt]{\small causal order violation = rip consistency} (5.5*\X, -2.5*\Y);
\end{tikzpicture}

%%% Local Variables:
%%% mode: latex
%%% TeX-master: "../paper"
%%% End:

  \caption{\label{fig:generalproblem}Broadcast without causal order
    enforcement.}
  \end{center}
\end{figure}

Figure~\ref{fig:generalproblem} depicts an example where causal order is
violated. Process $p_1$ broadcasts and delivers $m$. Process $p_2$ receives and
delivers $m$. Then, it broadcasts and delivers $m'$. Process $p_3$ receives $m'$
before $m$. Without any causal order enforcement, $p_3$ delivers $m'$ before $m$
violating the condition stating that the delivery of $m$ should precede the
delivery $m'$.

Multiple approaches exist to enforce causal order by either piggybacking control
information in each message, or constraining the network topology. In this
paper, we introduce a causal broadcast protocol freed from both these
requirements.


%%% Local Variables:
%%% mode: latex
%%% TeX-master: "../paper"
%%% End:
