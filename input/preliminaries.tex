
\section{Preliminaries}
\label{sec:preliminaries}

Definitions and theorems come from~\cite{hadzilacos1994modular}.

\begin{definition}[Network]
  A network $N$ comprises a set of processes $P$ and a set of edges
  $E: P \times P$. Each process runs a set of instructions
  sequentially. Processes linked by an edge can communicate with each
  other. Processes are faulty if they crash, otherwise they are correct. The set
  of correct processes is $C$.
\end{definition}

\TODO{Replace ``network'' by  ``distributed system'' ?}
\TODO{Add Non-byzantine.}


\begin{definition}[Happens before $\rightarrow$~\cite{lamport1978time}]
  Happens before is a transitive, irreflexive, and antisymmetric relation that
  defines a strict partial orders of events. At process $p$, Event $e$ happens
  before -- or precedes -- Event $e'$ is noted $e_p \rightarrow e'_p$. The
  sending of a message always precedes its receipt: \\
  $\forall p,\,q \in P, s_p(m) \rightarrow r_q(m)$.
\end{definition}

\begin{definition}[Uniform reliable broadcast]
  A process $p$ broadcasts a message $m$ to all processes of the network: \\
  $\forall p \in P,\, (b_p(m) \Leftrightarrow \forall q \in P,\, r_q(m))$. \\ URB guarantees 3 properties: 

  \begin{asparadesc}
  \item [Validity:] If a correct process broadcasts a message, then it eventually
    delivers it: $\forall p \in C,\, b_p(m) \rightarrow d_p(m)$.
  \item [Uniform Agreement:] If a process -- correct or not -- delivers a message,
    then all correct processes eventually deliver it:\\
    $\forall p \in P,\, (d_p(m) \implies \forall q \in C,\, d_q(m))$.
  \item [Uniform Integrity:] A process delivers a message at most once, and
    only if it was previously broadcast:\\
    $\forall p \in P,\, \neg(d_p(m) \rightarrow d_p(m)) \wedge$\\$d_p(m)
    \implies \exists q \in P,\, b_q(m) \rightarrow d_p(m)$.
\end{asparadesc}

\end{definition}


\begin{definition}[FIFO order]
  If a process broadcasts two messages, processes deliver the first before the
  second:\\
  $\forall p,\,q \in P,\,$\\$b_p(m) \rightarrow b_p(m') \implies d_q(m) \rightarrow
  d_q(m')$.
\end{definition}

\begin{definition}[Local order]
  If a process broadcasts a message after having delivered another message
  broadcast by another process, processes deliver the later before the former:\\
  $\forall p,\,q,\,r,\, \in P,\,p\neq q,\,$\\$b_p(m) \wedge d_q(m) \rightarrow b_q(m') \implies d_r(m) \rightarrow d_r(m')$.
\end{definition}

\begin{definition}[Causal order]
  The delivery order of messages follows the happen before relationships of the
  corresponding broadcasts:\\ $\forall
  p,\,q,\,r \in P,\,$\\$b_p(m) \rightarrow b_q(m') \implies d_r(m) \rightarrow d_r(m')$.
\end{definition}

\begin{theorem}[\label{theo:causal}Causal order equivalence]
  The intersection of FIFO order and Local order is equivalent to Causal order.
\end{theorem}

\begin{figure}
  \begin{center}
  
\begin{tikzpicture}[scale=1]
  \small

  \newcommand\X{35pt};
  \newcommand\Y{30pt};
  
  \draw[->](0pt,   0pt)--(6*\X,   0pt);
  \draw[->](0pt, -1*\Y)--(6*\X, -1*\Y);
  \draw[->](0pt, -2*\Y)--(6*\X, -2*\Y);
  
  \draw[fill=black](0pt, 0pt) node[anchor=east]{$p_1$}circle(2pt);
  \draw[fill=black](0pt, -1*\Y) node[anchor=east]{$p_2$}circle(2pt);
  \draw[fill=black](0pt, -2*\Y) node[anchor=east]{$p_3$}circle(2pt);

  \draw[->](0.5 * \X, 0pt)node[anchor=south]{$ins(a_{1,\,1})$}--(1*\X,   -1 * \Y);
%  \draw[->](0.5 * \X, 0pt)-- (2*\X, -1*\Y) -- ( 5 * \X,   -2 * \Y);
  \draw[->](0.5 * \X, 0pt) to[out=-45, in=90+60] ( 5 * \X,   -2 * \Y);
  \draw(1.5*\X, -2pt)--node[anchor=south]{$\{a_{1,\,1} \}$}(1.5*\X, 2pt);
  \draw(1.25 * \X, -2 -1*\Y )--node[anchor=north]{$\{a_{1,\,1} \}$}(1.25 * \X,    2 -1 * \Y);
  \draw(5.3 * \X, -2 -2*\Y )--node[anchor=north]{$\{a_{1,\,1}\}$}(5.3 * \X,   2 -2 * \Y);

  \draw[->](3.5 * \X, -1*\Y )--(4 * \X,   -2 * \Y);
  \draw[->](3.5 * \X, -1*\Y )node[anchor=south east]{$rem(a_{1,\,1})$}--(4 * \X,    0 * \Y);
  \draw(4.5 * \X, -2 -0*\Y )--node[anchor=south]{$\{\,\}$}(4.5 * \X,    2 -0 * \Y);
  \draw(4.5 * \X, -2 -1*\Y )--node[anchor=south]{$\{\,\}$}(4.5 * \X,    2 -1 * \Y);
  \draw(4.5 * \X, -2 -2*\Y )--node[anchor=north]{$\{\,\}$}(4.5 * \X,    2 -2 * \Y);

  % \draw[decorate,decoration={brace,amplitude=6pt,mirror,raise=4pt}]
  % (4.5*\X, -2.5*\Y) -- node[anchor=north, yshift=-10pt]{\small causal order violation = rip consistency} (5.5*\X, -2.5*\Y);
\end{tikzpicture}

%%% Local Variables:
%%% mode: latex
%%% TeX-master: "../paper"
%%% End:

  \caption{\label{fig:generalproblem}Broadcast without causal order
    enforcement.}
  \end{center}
\end{figure}

Figure~\ref{fig:generalproblem} depicts an example where causal order is
violated. Process $p_1$ broadcasts and delivers $m$. Process $p_2$ receives and
delivers $m$. Then, it broadcasts and delivers $m'$. Process $p_3$ receives $m'$
before $m$. Without any causal order enforcement, $p_3$ delivers $m'$ before $m$
violating the condition stating that the delivery of $m$ should precede the
delivery $m'$.

\begin{definition}[Causal broadcast]
  Causal broadcast is a uniform reliable broadcast ensuring causal order.
\end{definition}

Multiple approaches exist to enforce causal order by either piggybacking control
information in each message, or constraining the network topology. In this
paper, we introduce a causal broadcast protocol freed from both these
requirements.


%%% Local Variables:
%%% mode: latex
%%% TeX-master: "../paper"
%%% End:
