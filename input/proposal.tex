
\section{Causal broadcast}
\label{sec:proposal}

We propose an approach guaranteeing causal order to uniform reliable
broadcasts. It only makes use of FIFO communication channels (e.g. TCP), and
relies on scalable peer-sampling protocols that must not create network partitions
(e.g. \SPRAY~\cite{nedelec2017adaptive}, or \CYCLON~\cite{voulgaris2005cyclon}).

\paragraph{Static networks.} A network is a set of processes linked by a set of
channels. In a static network, processes cannot fail nor leave, and links cannot
be created nor removed. In this context, FIFO channels are sufficient to provide
causal ordering of messages.

\begin{figure}
  \begin{center}
    
\begin{tikzpicture}[scale=1]
  \small

  \newcommand\X{35pt};
  \newcommand\Y{30pt};
  
  \draw[->](0pt,   0pt)--(6*\X,   0pt);
  \draw[->](0pt, -1*\Y)--(6*\X, -1*\Y);
  \draw[->](0pt, -2*\Y)--(6*\X, -2*\Y);
  
  \draw[fill=black](0pt, 0pt) node[anchor=east]{$p_1$}circle(2pt);
  \draw[fill=black](0pt, -1*\Y) node[anchor=east]{$p_2$}circle(2pt);
  \draw[fill=black](0pt, -2*\Y) node[anchor=east]{$p_3$}circle(2pt);

  \draw[->](0.5 * \X, 0pt)node[anchor=south]{$b(m)$}--(2 * \X,   -1 * \Y); %% p1 -> p2 m
  \draw[->](2*\X, -1*\Y) -- (2.5*\X, 0*\Y); %% p2 -> p1 m
  \draw[->](2*\X, -1*\Y) -- (2.5*\X, -2*\Y); %% p2 -> p3 m

  \draw[->](0.5 * \X, 0pt)--( 5 * \X,   -2 * \Y); %% p1 -> p3 m

  \draw(1.5*\X, -2pt)--node[anchor=south]{$d(m)$}(1.5*\X, 2pt);
  \draw(2.25 * \X, -2 -1*\Y )--node[anchor=north]{$d(m)$}(2.25 * \X,    2 -1 * \Y);
  \draw(2.75 * \X, -2 -2*\Y )--node[anchor=north]{$d(m)$}(2.75 * \X,    2 -2 * \Y);

  \draw[->](3.5 * \X, -1*\Y )--(4 * \X,   -2 * \Y);
  \draw[->](3.5 * \X, -1*\Y )node[anchor=south east]{$b(m')$}--(4 * \X,    0 * \Y);
  \draw[->] (4*\X, 0*\Y) -- (5.5*\X, -1*\Y); %% p1 -> p2 m'
  \draw (4*\X, 0*\Y) -- (6*\X, -0.85*\Y); %% p1 -> p3 m'  

  \draw(4.5 * \X, -2 -0*\Y )--node[anchor=south]{$d(m')$}(4.5 * \X,    2 -0 * \Y);
  \draw(4.5 * \X, -2 -1*\Y )--node[anchor=south]{$d(m')$}(4.5 * \X,    2 -1 * \Y);
  \draw(4.5 * \X, -2 -2*\Y )--node[anchor=north]{$d(m')$}(4.5 * \X,    2 -2 * \Y);
  
  % \draw[decorate,decoration={brace,amplitude=6pt,mirror,raise=4pt}]
  % (4.5*\X, -2.5*\Y) -- node[anchor=north, yshift=-10pt]{\small causal order violation} (5.5*\X, -2.5*\Y);
\end{tikzpicture}

%%% Local Variables:
%%% mode: latex
%%% TeX-master: "../paper"
%%% End:

    \caption{\label{fig:static}FIFO channels are sufficient to provide causal
      order in static networks.}
  \end{center}
\end{figure}

Figure~\ref{fig:static} shows a static network comprising 3 processes $p_1$,
$p_2$, and $p_3$ linked to each other. The figure does not show messages sent by
$p_3$ for the sake of clarity. In this example, we solve the causal order
violation from Figure~\ref{fig:generalproblem} by using FIFO channels. The
processes receive $m$ and $m'$ multiple times but there exist no link in the
paths from $p_2$ to $p_3$ that carries $m'$ without having carried $m$
beforehand. Hence, the delivery of $m$ always precedes the delivery of $m'$ at
any process.

\paragraph{Dynamic networks.} Removing a channel does not impair causal ordering
of messages except if it generates network partitions.  Adding a channel may
result in causal ordering violation.


\begin{figure}
  \begin{center}
    
\begin{tikzpicture}[scale=1]
  \newcommand\X{190/6pt};
  \newcommand\Y{30pt};

  \draw[->](0pt,   0pt)--(6*\X,   0pt);
  \draw[->](0pt, -1*\Y)--(6*\X, -1*\Y);
  \draw[->](0pt, -2*\Y)--(6*\X, -2*\Y);
 
  \draw[fill=black](0pt, 0pt) node[anchor=east]{$p_1$}circle(2pt);
  \draw[fill=black](0pt, -1*\Y) node[anchor=east]{$p_2$}circle(2pt);
  \draw[fill=black](0pt, -2*\Y) node[anchor=east]{$p_3$}circle(2pt);


  \draw[->](0.5*\X, 0pt)node[anchor=south]{$b(m)$}--(3*\X, -1*\Y);
  \draw(1.25*\X, 2pt)--node[anchor=south]{$d(m)$}(1.25*\X, -2pt);
  \draw(3.25*\X, -2-1*\Y )--node[anchor=south]{$d(m)$}(3.25*\X, 2-1*\Y);

  \draw[->](3*\X, -1*\Y)--(5*\X, -2*\Y);
  \draw(5.25*\X, -2-2*\Y )--node[anchor=north]{$d(m)$}(5.25*\X, 2-2*\Y);

  \draw[fill=white](1.75*\X, 0pt) circle(2pt);
  \draw[fill=white](1.75*\X, -2*\Y) circle(2pt);

  \draw[->, dashed](1.75*\X, 0pt) -- (1.75*\X, -2.5*\Y)
  node[anchor=north]{$p_1$ connects to $p_3$};

  \draw[->](2.5*\X, 0pt)node[anchor=south]{$b(m')$} -- (2.75*\X, -2*\Y);
  \draw(3.5*\X, -2pt )--node[anchor=south]{$d(m')$}(3.5*\X, 2pt);
  \draw[->](2.5*\X, 0pt) -- (5*\X, -1*\Y);
  \draw(3.25*\X, -2-2*\Y )--node[anchor=north]{$d(m')$}(3.25*\X, 2-2*\Y);
  \draw(5.25*\X, -2-1*\Y )--node[anchor=south]{$d(m')$}(5.25*\X, 2-1*\Y);


  \draw[decorate,decoration={brace,amplitude=6pt,mirror,raise=4pt}] (3.25*\X,
  -2.5*\Y) -- node[anchor=north, yshift=-10pt]{\small causal order violation}
  (5.25*\X, -2.5*\Y);

  \draw(5*\X, -1*\Y)--(6*\X, -1.5*\Y);

  
\end{tikzpicture}

    \caption{\label{fig:problem}Adding a FIFO channel endangers causal
      ordering.}
  \end{center}
\end{figure}

Figure~\ref{fig:problem} illustrates the issue with the establishment of new
FIFO channels. In this example, a FIFO channel links $p_1$ to $p_2$; another
links $p_2$ to $p_3$; none links $p_1$ to $p_3$. Other FIFO channels may exist
but we do not show them for the sake of simplicity. Process $p_1$ broadcasts $m$
and delivers it. $p_3$ receives it by the intermediary of $p_2$. In the
meantime, $p_1$ creates a FIFO channel to $p_3$, then broadcasts $m'$ to $p_2$
and $p_3$. Since the path through $p_2$ is longer in terms of propagation time
compared to the direct connections from $p_1$ to $p_3$, Process $p_3$ receives
and delivers $m'$ before $m$. It violates causal order, for $m$ precedes $m'$.


Solving this issue requires three steps. When a process $p$ wants to use a link to
a process $q$ for causal broadcast,
\begin{enumerate}[(i)]
\item it must use the FIFO channels already in use to transmit a message to $q$
  awaiting for its acknowledgment.
\item While awaiting for this acknowledgement, $p$ buffers each message it
  sends.
\item When $p$ receives the acknowledgment from $q$, it uses the FIFO channel
  from $p$ to $q$ to send its buffered messages. Afterwards, the channel is
  ready to be used for causal broadcast.
\end{enumerate}

\begin{algorithm*}[h]
  
\algblockdefx[initially]{initially}{endInitially}
[0] {\textbf{INITIALLY:}}

\algsetblockdefx[event]{event}{endEvent}
{65535}{}
[0] {\textbf{EVENTS:}}

\algsetblockdefx[dissemination]{dissemination}{endDissemination}
{65535}{}
[0] {\textbf{DISSEMINATION:}}


\newcommand{\comment}[1]{$\rhd$ #1}
\newcommand{\LINEIFTHEN}[2]{%
  \algorithmicif\ {#1}\ \algorithmicthen\ {#2} %
}
\newcommand{\LINEIFTHENELSE}[3]{%
  \algorithmicif\ {#1}\ \algorithmicthen\ {#2} \algorithmicelse\ {#3}%
}

\newcommand{\LINEFOR}[2]{%
  \algorithmicfor\ {#1}\ \algorithmicdo\ {#2} %
}

\begin{algorithmic}[1]

  \initially
  \State $p$ \hfill \comment{This peer identity}
  \State $P$
  \hfill \comment{Partial view of the network, ie, a set of FIFO communication channels}
  \State $B \leftarrow \varnothing$ 
  \hfill \comment{Buffers of messages}
  \endInitially
  
  \event
  \Function{onChannelOpen} {$i$, $q$}
  \hfill \comment{$i$: the initiator, $q$: the channel}
  \If{$(i\neq p) \wedge |P|-|B|>1$}
  \hfill \comment{The channel is temporarily unsafe, for it may create ``shortcuts''}  
  \State $B[q] \leftarrow \varnothing$
  \State $\textsc{broadcast}(\langle p,\, q\rangle,\, ``locked")$
  \hfill \comment{Broadcast a ``locked'' message waiting $q$'s acknowledgment}
  \EndIf
  \EndFunction
  
  \Statex

  \Function{onChannelClose} {$q$} \hfill \comment{$q$: the channel that closed}
  \State $B \leftarrow B \setminus q$
  \EndFunction

  \endEvent

  \dissemination

  \Function{broadcast} {$payload$, $type$ [$\leftarrow``regular"$]} 
  \State \LINEFOR{\textbf{each} $q \in B$}
  {$B[q] \leftarrow B[q] \cup \langle type,\, payload\rangle$}
  \hfill{Buffer the message to send it later to peers in $B$}
  \State \LINEFOR{\textbf{each} $q \in P \setminus B$} 
  {$\textsc{sendTo}(q,\, \langle type,\, payload \rangle)$}
  \hfill{Broadcast the message to neighbors}
  \EndFunction

  \Statex

  \Function{receive} {$message$}

  \If {$\neg \textsc{alreadyReceived}(message)$}
  \hfill \comment{Check each message only once}

  \State \textbf{let} $\langle type,\, payload \rangle \leftarrow message$
  
  \If {$type = ``locked"$}
  \State \textbf{let} $\langle from,\, to \rangle \leftarrow payload$
  \State \LINEIFTHEN {$p = to$} 
  {$\textsc{sendTo}(from,\, \langle ``unlock",\, to \rangle)$}
  \hfill \comment{Notify the sender that we received the message}
  \State \LINEIFTHEN {$p \neq to$} {$\textsc{broadcast}(payload,\, type)$}
  \hfill \comment{Or forward it as a regular broadcast message}
  \EndIf
  
  \If {$type = ``unlock"$}
  \State \textbf{let} $to \leftarrow payload$
  \If {$to \in B$}
  \hfill \comment{Got notified, the channel to $to$ is now safe}
  \State \LINEFOR {\textbf{each} $b \in B[to]$}
  {$\textsc{sendTo}(to,\, b)$}
  \hfill \comment{Send all buffered message through this channel}
  \State $B \leftarrow B \setminus to$
  \hfill \comment{And start using it as a regular channel}
  \EndIf
  \EndIf
  
  \If {$type = ``regular"$} \hfill \comment{Classic forwarding of broadcast messages}
  \State $\textsc{deliver}(payload)$
  \State $\textsc{broadcast}(payload)$
  \EndIf
  \EndIf

  \EndFunction

\end{algorithmic}


%%% Local Variables:
%%% mode: latex
%%% TeX-master: "../paper"
%%% End:

  \caption{\label{algo:causalbroadcast}Causal broadcast.}
\end{algorithm*}

\begin{figure}
  \begin{center}
    
\begin{tikzpicture}[scale=1]
  \newcommand\X{210/6pt};
  \newcommand\Y{30pt};

  \draw[->](0pt,   0pt)--(6*\X,   0pt);
  \draw[->](0pt, -1*\Y)--(6*\X, -1*\Y);
  \draw[->](0pt, -2*\Y)--(6*\X, -2*\Y);
 
  \draw[fill=black](0pt, 0pt) node[anchor=east]{$p_1$}circle(2pt);
  \draw[fill=black](0pt, -1*\Y) node[anchor=east]{$p_2$}circle(2pt);
  \draw[fill=black](0pt, -2*\Y) node[anchor=east]{$p_3$}circle(2pt);


  \draw[->](0.5*\X, 0pt)node[anchor=south]{$b(m)$}--(2*\X, -1*\Y);
  \draw(1.15*\X, 2pt)--node[anchor=south]{$d(m)$}(1.15*\X, -2pt);
  \draw[->, dashed](1.75*\X, 0*\Y) -- (2.5*\X, -1*\Y);
%%  \draw(3.25*\X, -2-1*\Y )--node[anchor=south]{$d(m)$}(3.25*\X, 2-1*\Y);

%%  \draw[->](3*\X, -1*\Y)--(5*\X, -2*\Y);
%%  \draw(5.25*\X, -2-2*\Y )--node[anchor=north]{$d(m)$}(5.25*\X, 2-2*\Y);

  \draw[fill=white](1.5*\X, 0pt) circle(2pt);
  \draw[fill=white](1.5*\X, -2*\Y) circle(2pt);

  \draw[->, dashed](1.5*\X, 0pt) -- (1.5*\X, -2.5*\Y)
  node[anchor=north]{$p_1$ connects to $p_3$};

  % \draw[->](2.5*\X, 0pt)node[anchor=south]{$b(m')$} -- (2.75*\X, -2*\Y);
  % \draw(3.5*\X, -2pt )--node[anchor=south]{$d(m')$}(3.5*\X, 2pt);
  % \draw[->](2.5*\X, 0pt) -- (5*\X, -1*\Y);
  % \draw(3.25*\X, -2-2*\Y )--node[anchor=north]{$d(m')$}(3.25*\X, 2-2*\Y);
  % \draw(5.25*\X, -2-1*\Y )--node[anchor=south]{$d(m')$}(5.25*\X, 2-1*\Y);


  % \draw[decorate,decoration={brace,amplitude=6pt,mirror,raise=4pt}] (3.25*\X,
  % -2.5*\Y) -- node[anchor=north, yshift=-10pt]{\small causal order violation}
  % (5.25*\X, -2.5*\Y);

  %% \draw(5*\X, -1*\Y)--(6*\X, -1.5*\Y);

  
\end{tikzpicture}

    \caption{\label{fig:solved}Causal broadcast in dynamic networks.}
  \end{center}
\end{figure}


%%% Local Variables:
%%% mode: latex
%%% TeX-master: "../paper"
%%% End:
