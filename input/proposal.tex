
\section{Causal broadcast}
\label{sec:proposal}

We propose an approach guaranteeing causal order to uniform reliable
broadcasts. It only makes use of FIFO communication channels (e.g. TCP), and
relies on scalable peer-sampling protocols that must not create network partitions
(e.g. \SPRAY~\cite{nedelec2017adaptive}, or \CYCLON~\cite{voulgaris2005cyclon}).

\paragraph{Static networks.} A network is a set of processes linked by a set of
channels. In a static network, processes cannot fail nor leave, and links cannot
be created nor removed. In this context, FIFO channels are sufficient to provide
causal ordering of messages.

\begin{figure}
  \begin{center}
    \input{./input/figstatic.tex}
    \caption{\label{fig:static}FIFO channels are sufficient to provide causal
      order in static networks.}
  \end{center}
\end{figure}

Figure~\ref{fig:static} shows a static network comprising 3 processes $p_1$,
$p_2$, and $p_3$ linked to each other. The figure does not show messages sent by
$p_3$ for the sake of clarity. In this example, we solve the causal order
violation from Figure~\ref{fig:generalproblem} by using FIFO channels. The
processes receive $m$ and $m'$ multiple times but there exist no link in the
paths from $p_2$ to $p_3$ that carries $m'$ without having carried $m$
beforehand. Hence, the delivery of $m$ always precedes the delivery of $m'$ at
any process.

\begin{theorem}[FIFO-based causal broadcast in static networks]
  \TODO{blablabla.}
\end{theorem}

\begin{proof} To prove causal ordering, we must show that such broadcast
provides both FIFO ordering and Local ordering of messages.
\begin{asparadesc}
\item [FIFO] Suppose a process $p$ broadcasts $m$ before $m'$. Consider that a
  correct process $q$ delivers $m'$. We must show that $q$ delivers $m$ before
  $m'$. Process $p$ broadcast message $m$ before broadcasting $m'$ to all its
  neighborhood using FIFO channels. Each neighbor eventually receives and
  delivers $m$ then $m'$. Each neighbor forwards each message exactly once in
  the order they delivered them to all its neighbors using FIFO channels. Since
  we assume a network without clusters, messages reach all processes. Process
  $q$ eventually receives and delivers $m$ first, then it eventually receives
  and delivers $m'$.
\item [Local] Suppose a process $p$ delivers $m$ before broadcasting
  $m'$. Consider a correct process $q$ that delivers $m'$. We must show that $q$
  delivers $m$ before $m'$. Since $p$ delivers $m$ before broadcasting $m'$, it
  forwards $m$ to all its neighbors before broadcasting $m'$ to the same
  neighbors. Similarly to FIFO ordering, due to the use of FIFO channels, the
  forwardings, and the absence of network clusters, Process $q$ eventually
  receives or already received $m$ when it receives $m'$. Thus, it delivers $m$
  before delivering $m'$.
\item [Causal] From Theorem~\ref{theo:causal}, since FIFO-based broadcast ensure
  both FIFO ordering and Local ordering, it ensures Causal ordering.
\end{asparadesc}
\end{proof}




\paragraph{Dynamic networks.} Removing a channel does not impair causal ordering
of messages except if it generates network partitions. \TODO{Theorem.} Adding a
channel may result in causal ordering violation. \TODO{Theorem.}


\begin{figure}
  \begin{center}
    \input{./input/figproblem.tex}
    \caption{\label{fig:problem}Adding a FIFO channel endangers causal
      ordering.}
  \end{center}
\end{figure}

Figure~\ref{fig:problem} illustrates the issue with the establishment of new
FIFO channels. In this example, a FIFO channel links $p_1$ to $p_2$; another
links $p_2$ to $p_3$; none links $p_1$ to $p_3$. Other FIFO channels may exist
but we do not show them for the sake of simplicity. Process $p_1$ broadcasts $m$
and delivers it. $p_3$ receives it by the intermediary of $p_2$. In the
meantime, $p_1$ creates a FIFO channel to $p_3$, then broadcasts $m'$ to $p_2$
and $p_3$. Since the path through $p_2$ is longer in terms of propagation time
compared to the direct connections from $p_1$ to $p_3$, Process $p_3$ receives
and delivers $m'$ before $m$. It violates causal order, for $m$ precedes $m'$.


Solving this issue requires three steps. When a process $p$ wants to use a link to
a process $q$ for causal broadcast,
\begin{enumerate}[(i)]
\item it must use the FIFO channels already in use to transmit a message to $q$
  awaiting for its acknowledgment.
\item While awaiting for this acknowledgement, $p$ buffers each message it
  sends.
\item When $p$ receives the acknowledgment from $q$, it uses the FIFO channel
  from $p$ to $q$ to send its buffered messages. Afterwards, the channel is
  ready to be used for causal broadcast.
\end{enumerate}

\begin{algorithm*}[h]
  
\algblockdefx[initially]{initially}{endInitially}
[0] {\textbf{INITIALLY:}}

\algsetblockdefx[event]{event}{endEvent}
{65535}{}
[0] {\textbf{EVENTS:}}

\algsetblockdefx[dissemination]{dissemination}{endDissemination}
{65535}{}
[0] {\textbf{DISSEMINATION:}}


\newcommand{\comment}[1]{$\rhd$ #1}
\newcommand{\LINEIFTHEN}[2]{%
  \algorithmicif\ {#1}\ \algorithmicthen\ {#2} %
}
\newcommand{\LINEFOR}[2]{%
  \algorithmicfor\ {#1}\ \algorithmicdo\ {#2} %
}

\begin{algorithmic}[1]

  \initially
  \State $p$ \hfill \comment{This peer identity}
  \State $P \leftarrow \varnothing$
  \hfill \comment{Partial view of the network, ie, a set of FIFO communication channels}
  \State $B \leftarrow \varnothing$ 
  \hfill \comment{Buffers of messages: $P\rightarrow M^*$ }
  \State $O \leftarrow \varnothing$
  \hfill \comment{Set of channels that just opened but are ``causaly'' unsafe.}
  \endInitially
  
  \event
  \Function{onChannelOpen} {$i$, $q$} \hfill \comment{$i$: the initiator, $q$: the channel}
  \State \LINEIFTHEN{$i \neq p$}{$O \leftarrow O \uplus q$}
  \EndFunction

  \Statex

  \Function{onChannelClose} {$q$} \hfill \comment{$q$: the channel that closed}
  \State $P \leftarrow P \setminus q$
  \State $B \leftarrow B \setminus q$
  \State $O \leftarrow O \setminus q$
  \EndFunction

  \endEvent

  \dissemination

  \Function{broadcast} {$m$} \hfill \comment{$m$: the message to send}
  \State $m \leftarrow m \cup \langle p, O\rangle$
  \hfill \comment{Mark the message with unsafe channels}
  \State $O \leftarrow \varnothing$
  \State \LINEFOR{\textbf{each} $b \in B$} {$b \leftarrow b \cup m$}
  \hfill{Buffer the message to send it later to peers in $B$}
  \State \LINEFOR{\textbf{each} $q \in P \setminus B$} 
  {$\textsc{sendTo}(q,\, m)$} \hfill{Broadcast the message to neighbors}
  \EndFunction

  \Statex

  \Function{receive} {$q$, $m$}
  \hfill \comment{$q$: the peer that sends the message, $m$: the received message}
  \For {\textbf{each} $\langle o,\, Q \rangle$}
  \hfill \comment{Notify the origin $o$ that we receive this particular message}
  \State \LINEIFTHEN {$p \in Q$} {$\textsc{sendTo}(o, \, m)$}
  \EndFor
  
  \If {($q \in B$) $\wedge$ ($m \in B[q]$)}
  \hfill \comment{Got notified, the channel to $q$ is now safe}
  \State \LINEFOR {\textbf{each} $n \in B[q]$} {$\textsc{sendTo}(q,\, n)$}
  \State $B \leftarrow B \setminus q$
  \EndIf
  
  \If {$\neg \textsc{alreadyReceived}(m)$}
  \hfill \comment{Forward the message to neighbors}
  \State $\textsc{deliver}(m)$
  \State $\textsc{broadcast}(m)$
  \EndIf

  \EndFunction

\end{algorithmic}


%%% Local Variables:
%%% mode: latex
%%% TeX-master: "../paper"
%%% End:

  \caption{\label{algo:causalbroadcast}Causal broadcast.}
\end{algorithm*}

\begin{figure}
  \begin{center}
    \input{./input/figsolved.tex}
    \caption{\label{fig:solved}Causal broadcast in dynamic networks.}
  \end{center}
\end{figure}

Figure~\ref{fig:solved} shows how our algorithm solves causal order
violations. Similarly to Figure~\ref{fig:problem}, a FIFO channel links $p_1$ to
$p_2$; another link $p_2$ to $p_3$; none links $p_1$ to $p_3$. Process $p_1$
broadcasts $m$. Then $p_1$ wishes to create a link to $p_3$. It sends a message
to $p_3$ using its FIFO channels. Thus, this message will follow $m$ at
$p_3$. When $p_3$ acknowledges this message to $p_1$, the later knows that it
receives at least $m$.


%%% Local Variables:
%%% mode: latex
%%% TeX-master: "../paper"
%%% End:
