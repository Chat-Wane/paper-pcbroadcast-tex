
\begin{abstract}
  Designing applications over a message-passing system goes through the
  definition and the implementation of appropriate communication primitives such
  as reliable broadcast , causal broadcast and atomic broadcast. This paper
  presents a new uniform reliable causal broadcast protocol. This is not one
  more protocol as the searching for an effective protocol is still an open
  problem especially in large and dynamic networks. The proposed protocol
  outperforms state-of-the-art approaches in size of messages, execution time
  complexity, and local space complexity. Most importantly, the control
  information piggybacked on messages is of constant size. We prove that for
  both static and dynamic networks. Consequently, large and dynamic systems can
  finally afford causal broadcast.

----------------

  Many distributed protocols and applications rely on causal broadcast to ensure
  consistency criteria.
  %% However, tracking causality while scaling remains an open
  %% problem. 
  However, none of causality tracking state-of-the-art approaches scale in large
  and dynamic systems.   
  %% State-of-the-art approaches do not scale in large and
  %% dynamic networks.
  % State-of-the-art approaches overload messages with control information the
  % size of which increases linearly compared to either the number of processes,
  % or the number of broadcast messages.
  % These approaches are impracticable in large scale systems subject to churn.
  We present a uniform reliable causal broadcast protocol that outperforms
  state-of-the-art in size of messages, execution time complexity, and local
  space complexity. Most importantly, it only overloads messages with constant
  size control information.  We prove that this holds in both static and dynamic
  systems.  Consequently, large and dynamic systems can finally afford causal
  broadcast.
\end{abstract}




%\keywords{Uniform, reliable, causal, broadcast.}

% \begin{IEEEkeywords}
% Uniform, reliable, causal, broadcast.
% \end{IEEEkeywords}

%%% Local Variables:
%%% mode: latex
%%% TeX-master: "../paper"
%%% End:
